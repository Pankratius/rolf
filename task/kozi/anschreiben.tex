\documentclass[a4paper]{article}
\usepackage{exercise}
%um nur aufgaben zu zeigen
%\usepackage[noanswer]{exercise} 
\usepackage{../images/preamble}


\begin{document}
	\vspace*{-2cm}
	\parbox{4cm}{\includegraphics[width=2.5cm]{../images/ROLF4.png}}
	\parbox{10.6cm}{\setstretch{2.0} \centering{ \huge \textsf{Der ROLF-Aufgabenwettbewerb }}\\  \href{mailto:physikrolf@gmail.com}{physikrolf@gmail.com} \\ \vspace*{-.5cm} }
\thispagestyle{empty}	
\vspace{1cm}
\noindent
\newline
\textsc{Willkommen} beim ROLF-Aufgabenwettbewerb!\\
Wenn ihr das lest, habt ihr vermutlich schon die erste Aufgabenserie versucht und hoffentlich Lösungen abgegeben\footnote{sonst: schnell eine E-Mail an  \href{mailto:physikrolf@gmail.com}{physikrolf@gmail.com} schreiben!} Das ist sehr gut!\\
Ab jetzt wird es monatlich von uns drei Aufgaben geben, die man lösen kann, ohne großes Vorwissen (sowie physikalisches als auch mathematisches) zu haben.  Die Aufgaben könnt ihr dann wieder bearbeiten, und bei uns (persönlich oder per Mail) abgeben, sodass wir sie bewerten können.\\
Jede Aufgabe hat eine geheime Anzahl an Punkten, die wir erst festlegen, wenn wir die Aufgaben bewerten. Was aber gleich bleibt, ist ein festes, prozentuales Bewertungsschema, mit dem wir jede Lösung einschätzen. Das beinhaltet alle Aspekte, die wir für wichtig beim Aufgabenlösen halten. Der eine große Teil der Bewertung ist der (richtige) Ansatz zum Lösen der Aufgabe. Hier wird klar, ob ihr die Aufgabenstellung und das Problem verstanden habt, und ob ihr eine Idee habt, wie man die Aufgabe lösen kann. Der andere große Teil der Bewertung ist die Argumentation zur Lösung der Aufgabe. Hier sieht man, ob ihr mit euren Ansätzen auch Lösungen finden könnt. 
\newline
\begin{figure}[h]
	\centering
\begin{tabular}{l| l c}
	Ansatz: & & 40\%\\\hline
	& nachvollziehbar:&20\%\\
	&zielführend:&40\%\\
	&math. Formulierung :& 40\%\\\hline\hline
	Argumentation:& &60\% \\\hline
	&nachvollziehbar:&40\%\\
	&richtig:&60\%\\\hline
	
\end{tabular}
\end{figure}
Die Punktzahlen sollen dafür stehen, wie schwer die Aufgaben sind. Zusätzlich kann es sein, dass eine (falsche) Annahme das Problem sehr vereinfacht. Das wird sich in einem prozentuellen Abzug von der Gesamtpunktzahl pro Aufgabe niederschlagen.\\
An dieser Stelle ein Paar Tipps, die dafür sorgen, dass wir bessere Laune haben, wenn wir eure Aufgaben korrigieren:
\begin{itemize}
	\item Einheiten sind eure Freunde! - Wenn man eine Temperatur und einen Widerstand addiert, eine Fläche und eine Masse, ist das ein gutes Zeichen dafür, dass man etwas falsch macht.
	\item Termumformungen können lang und hässlich werden! - Schreibt am Besten \underline{alle} Gleichungen auf, mit denen ihr denkt, die Aufgabe lösen zu können, bevor ihr mit langen Termumformungen anfangt.
	\item Aufgabenstellungen sind nicht zum Spaß da! - Wenn ihr gegebene Größen oder Annahmen aus der Aufgabenstellung nicht braucht, seid ihr entweder wirklich schlau oder habt die Aufgabe nicht richtig gelöst/verstanden.
\end{itemize}
Wenn ihr die Aufgaben gelöst habt, könnt ihr die Lösungen immer per Mail an \textbf{ \href{mailto:physikrolf@gmail.com}{physikrolf@gmail.com}} senden. Oder per Post an Aaron Wild, Zum wilden Graben 26, 99425 Weimar. Oder ihr gebt sie dem, der euch die Aufgaben gegeben hat. An den Tagen, an denen ihr die Aufgaben abgeben solltet (gerne auch früher, etwa 1x im Monat) bekommt ihr gleich die Musterlösungen für diese Serie und die nächsten Aufgaben. Sobald wir fertig mit dem Korrigieren sind, gibt es auch eine Rückmeldungen.\\
Insgesamt hoffen wir, dass ihr mit den Aufgaben ein wenig Spaß habt, und vielleicht sogar etwas dazu lernt. Feedback dazu, wie gut die Aufgaben sind, könnt ihr gerne mit abgeben!
\end{document}