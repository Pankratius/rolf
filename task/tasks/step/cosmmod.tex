\begin{Exercise}[label = cosmmod, origin = {STEP 2001, Paper 1}, title = {Hubbles  "Konstante"}, difficulty = 2]
	Man geht davon aus, dass der Radius $R\left(t\right), t\geq 0$ des Universums drei Bedingungen gehorcht
	\begin{subequations}\label{cosmmod:cond}
		\begin{equation}\label{cosmmod:bb}
			R\left(0\right) = 0
		\end{equation}
		\begin{equation}\label{cosmmod:rd}
			\dot{R} > 0 ~t>0
		\end{equation}
		\begin{equation}\label{cosmmod:rdd}
			\ddot{R} > 0 ~t>0.
		\end{equation}
	\end{subequations}
	Die Hubble-Funktion $H\left(t\right)$ ist definiert als
	\begin{equation}
		H\left(t\right) = \frac{\dot{R}}{R}.
	\end{equation}
	\Question Man hat beobachtet, dass $H\left(t\right)= \frac{a}{t}$, wobei $a$ eine Konstante ist. Berechne daraus einen Ausdruck für $R\left(t\right)$. Welche Werte darf $a$ haben, damit die Bedinungen \eqref{cosmmod:cond} weiter erfüllt bleiben?
	\Question Ist es möglich, dass $H\left(t\right) = \frac{b}{t^2}$, wobei $b$ eine Konstante ist? Vergleiche dazu mit den Bedingungen \eqref{cosmmod:cond}.
\end{Exercise}