\documentclass[a4paper]{article}
%\usepackage{exercise}
%um nur aufgaben zu zeigen
\usepackage[noanswer]{exercise} 
\usepackage{../images/preamble}


\begin{document}
	\vspace*{-2cm}
	\parbox{4cm}{\includegraphics[width=2.5cm]{../images/ROLF4.png}}
	\parbox{10.6cm}{\setstretch{2.0} \centering{ \huge \textsf{Analysis mit Physik üben III}}\\ \href{mailto:physikrolf@gmail.com}{physikrolf@gmail.com}, \url{pankratius.github.io/rolf} \\ \vspace*{-.5cm} }
	\begin{framed}
		\small
	Wir verwenden die Physikkonvention, eine zeitliche Ableitung als Punkt darzustellen; $\dot{x} = \frac{dx\left(t\right)}{dt} = \frac{d}{dt}\left(x\left(t\right)\right)$.
	\end{framed}
	

\thispagestyle{empty}


\noindent
\begin{Exercise}[label = cosmmod, origin = {STEP 2001, Paper 1}, title = {Hubbles  "Konstante"}, difficulty = 2]
	Man geht davon aus, dass der Radius $R\left(t\right), t\geq 0$ des Universums drei Bedingungen gehorcht
	\begin{subequations}\label{cosmmod:cond}
		\begin{equation}\label{cosmmod:bb}
			R\left(0\right) = 0
		\end{equation}
		\begin{equation}\label{cosmmod:rd}
			\dot{R} > 0 ~t>0
		\end{equation}
		\begin{equation}\label{cosmmod:rdd}
			\ddot{R} > 0 ~t>0.
		\end{equation}
	\end{subequations}
	Die Hubble-Funktion $H\left(t\right)$ ist definiert als
	\begin{equation}
		H\left(t\right) = \frac{\dot{R}}{R}.
	\end{equation}
	\Question Man hat beobachtet, dass $H\left(t\right)= \frac{a}{t}$, wobei $a$ eine Konstante ist. Berechne daraus einen Ausdruck für $R\left(t\right)$. Welche Werte darf $a$ haben, damit die Bedinungen \eqref{cosmmod:cond} weiter erfüllt bleiben?
	\Question Ist es möglich, dass $H\left(t\right) = \frac{b}{t^2}$, wobei $b$ eine Konstante ist? Vergleiche dazu mit den Bedingungen \eqref{cosmmod:cond}.
\end{Exercise}
\begin{Exercise}[label = frosty, origin = {STEP 1991, 1. Paper}, title = {Schneemann}, difficulty = 2]
	Ein Schneemann besteht aus zwei Schneebällen, mit Radii $R$, $2R$ und $3R$.\\
	Jetzt fängt der Schneemann an, zu schmelzen. Finde, mit geeigneten Annahmen, das Verhältnins von Anfangsvolumen zum Volumen zu dem Zeitpunkt, an dem sich die Gesamthöhe des Schneemanns halbiert hat.
\end{Exercise}
\begin{Exercise}[label = snowploughing, title = Schneepflug, origin = {STEP 1987, Paper 3}, difficulty = 4]
	Zwei identische Schneepflüge räumen die gleiche Straße. Der erste startet eine Zeit $t_1$ nach dem es mit scheinen anfing, der zweite vom gleichen Punkt nach einer Zeit $t_2-t_1$.\\
	Der Schnee fällt so, dass sich die Höhe der Schneedecke mit einer kontanten Rate $k$ vergrößert.\\
	Die Geschwindigkeit einer Schneeraupe ist $v\left(t\right) = a k/z\left(t\right)$, wobei $z\left(t\right)$ die momentante Schneehöhe angibt, und $a$ eine Konstante ist.\\
	Jeder Schneepflug räumt den gesamten Schnee. Zeige, dass die Zeit $t$, zu der der zweite Schneepflug eine Distanz $x_2\left(t\right)$ zurückgelegt hat, die Gleichung
	\begin{equation}\label{snowp:invertedeom}
		a \frac{dt}{dx_2} = t - t_1e^{\left(\nicefrac{x_2}{a}\right)}
	\end{equation}
	erfüllt.\\
	Bestimme mit \eqref{snowp:invertedeom} die Zeit, bis die beiden Schneepflüge kollidieren.
\end{Exercise}



\end{document}
