\documentclass[a4paper]{article}
\usepackage[utf8]{inputenc}
\usepackage[german]{babel}
\usepackage{amsmath}
\usepackage{amsthm}
\usepackage{amsfonts}
\usepackage{amssymb}
\usepackage{graphicx}
\usepackage{wrapfig}
\usepackage{float}
\usepackage{tkz-euclide}
\usetkzobj{all}
\usepackage{tikz}
\usepackage{exercise}
\usepackage{import}
\usepackage{hyperref}
\usepackage[european]{circuitikz}

\usepackage{framed}
\usepackage{setspace} %Für \setstretch{}, welches Zeilenabstände reguliert.

\usepackage{caption}
\usepackage{subcaption}


\usepackage[left=2cm,right=2.5cm,top=2.5cm,bottom=2cm]{geometry}

\usepackage{titlesec}
\numberwithin{equation}{Exercise}


  \renewcommand{\ExerciseHeader}{\flushleft\textsf{\textbf{
               Aufgabe \ExerciseHeaderNB} (\ExerciseHeaderTitle)\smallskip\newline
               }}
 \renewcommand{\AnswerHeader}{\flushright \ExerciseHeaderDifficulty\textbf{\textsf{Lösung
              \ExerciseHeaderNB} }\ExerciseHeaderOrigin\smallskip\newline}

\renewcommand{\ExerciseHeaderTitle}{\ExerciseTitle}
 \setlength{\ExerciseSkipBefore}{0\baselineskip}


\begin{document}
	\vspace*{-2cm}
	\parbox{4cm}{\includegraphics[width=2.5cm]{/Users/aaron/Desktop/rolf/images/ROLF4.png}}
	\parbox{10.6cm}{\setstretch{2.0} \centering{ \huge \textsf{Aufgabenserie 1}}\\ Abgabe: 19. Mai \\ \vspace*{-.5cm} }
	
	

\thispagestyle{empty}
\begin{framed}
	\noindent
	\scriptsize
	Die Aufgaben sollten bis zum \textbf{19. Mai} bearbeitet werden. Die Lösungen schickt ihr entweder an \href{mailto:physikrolf@gmail.com}{physikrolf@gmail.com}, oder per Post/Brieftaube an: Dominos Pizza, Rollgasse 13, 99423 Weimar.
	Jede Aufgabe hat eine bestimmte Anzahl an erreichbaren Punkten. Wie viele dass sind, müsst ihr raten. Versucht, die Lösungen so genau wie möglich aufzuschreiben. Für besonders schnelle/witzige Lösungen kann es Bonuspunkte geben.
\end{framed}

\noindent

	\begin{Exercise}[label=ipho201741, title = Eisenbahnwaggon,difficulty={2},origin = {4. Runde IPhO 2017, Kurzaufgabe}]
 Ein Eisenbahnwaggon der Masse $m_1 = 37~\mathrm{t}$ rollt reibungsfrei auf einem Bahngleis und stößt mit einem anderen Waggon zusammen. Beide verkuppeln sich und rollen zusammen weiter. Dabei geben sie 35\% der anfangs vorhandenen kinetischen Energie als Wärme ab. Wie groß ist die Masse $m_2$ des zweiten Waggons?
	\end{Exercise}
\begin{Answer}[ref = ipho201741]
	Da keine äußeren Kräfte wirken, bleibt der Impuls erhalten.\\ Gleichzeitig bleibt die Gesamtenergie erhalten,
	\begin{equation}\label{energycons}
		E_{kin,v} = E_{kin,n}+E_{w}.
	\end{equation}
	Hierbei ist $E_{kin,v}$ die kinetische Energie vor dem Stoß, $E_{kin,n}$ die kinetische Energie nach dem Stoß, und $E_{w}$ die abgegebene Wärmeenergie. Es ist gegeben, dass $E_{w} = \eta \cdot E_{kin,v}$ gilt, wobei $\eta = 0.35$ gilt. Um die Impulserhaltung zu benutzten kann man entweder die kinetische Energie durch die Impulse oder aber durch die Geschwindigkeiten ausdrücken.\\
	Drückt man die kinetische Energie durch die Impulse aus, so ist im Allgemeinen $E_{kin} = \frac{p^2}{2m}$. Da der Impulserhalten bleibt, gilt $p:=p_v = p_n = \mathrm{konst.}$. Damit lässt sich die Energieerhaltung schreiben als
	\begin{equation}\label{penenergycons}
		\frac{p^2}{2m_1}= \frac{p^2}{2\left(m_1+m_2\right)} + \eta\cdot \frac{p^2}{2m_1},
	\end{equation}
	da nach dem Stoß nur noch ein \glqq Waggon\grqq{} der Masse $m_1+m_2$ da ist.
	Division durch $p^2$ auf beiden Seiten führt auf
	\begin{equation}\label{masscond}
		\frac{1}{2m_1}= \frac{1}{2\left(m_1+m_2\right)}+\frac{\eta}{2m_1}.
	\end{equation}
	Diese Gleichung erhält nur noch die bekannten Größen $m_1$ und $\eta$, sowie die unbekannte Größe $m_2$. Subtrahieren von $\frac{\eta}{2m_1}$, multiplizieren mit 2 und anschließendes Bilden des Reziproken führt auf
	\begin{equation*}\label{solp}
	\boxed{
		\frac{1}{m_1}\left(1-\eta\right) = \frac{1}{m_1+m_2} \Rightarrow m_2 = m_1\left(\frac{1}{1-\eta}-1\right)\approx 20~\mathrm{t}.}
	\end{equation*}
	Das gleiche kann man auch mit den Geschwindigkeiten machen. Hier hat die Energieerhaltung die Form
	\begin{equation}\label{venergycons}
		m_1 \frac{v_v^2}{2} = \left(m_1+m_2\right)\frac{v_n^2}{2} + m_1\eta  \cdot \frac{v_v^2}{2}.
	\end{equation}
	Explizit ausgeschrieben besagt die Impulserhaltung, dass 
	\begin{equation}\label{secondvel}
		m_1 v_v = \left(m_1+m_2\right)v_n \Rightarrow v_n = \frac{m_1}{m_1+m_2}v_v
	\end{equation}
	gilt. Einsetzen von \eqref{secondvel} in \eqref{venergycons} führt auf
	\begin{equation*}
	m_1\frac{v_v^2}{2} = \left(m_1+m_2\right)\frac{v_v^2 m_1^2}{2\left(m_1+m_2\right)^2} + m_1\eta \cdot \frac{v_v^2}{2}.
	\end{equation*}
	Dividieren durch $m_1^2v_v^2$ führt auf 
	\begin{equation}\label{vmasscond}
		\frac{1}{2m_1} = \frac{1}{2\left(m_1+m_2\right)} + \frac{\eta}{2m_1}.
	\end{equation}
	Die Bedingung, die die Massen erfüllen müssen, wenn wir Geschwindigkeiten betrachten (Gleichung \eqref{vmasscond}) entspricht also genau der, die erfüllt sein muss, wenn wir Impulse betrachten (Gleichung \eqref{masscond}). Das sollte auch so sein.
\end{Answer}

\begin{Exercise}[label = cards, title = Kartenhäuser, difficulty = 4]
		Ein Kartenhaus besteht im Grunde aus Zellen, die alle die Form haben, wie sie in Abb. \ref{fig:cells} gezeigt ist.
	\begin{itemize}
		\item[a)] Zeige, dass die Anzahl $A$ der Karten, die für ein solches Kartenhaus mit $n$ Ebenen benötigt wird,
		\begin{equation}\label{number}
		\frac{1}{2}n\left(3n+1\right)
		\end{equation}
		beträgt.
		
		\item[b)] 
		Betrachte nun zuerst zwei Karten, auf die eine Kraft $\vec{F}$ von oben wirkt. Bestimme den Reibungskoeffizienten $\mu$ zwischen Karte und Boden, wenn der Anstellwinkel der Karten $\theta = 60^\circ$ beträgt und die Masse einer Karte $m$ ist.
		
		\item[c)] Wenn man mehr als eine Kartenebene hat, muss man auch die Reibung zwischen den Kartenspitzen und der daraufliegenden Deckkarte $\nu$ betrachten. Wie groß muss $\nu$ mindestens sein, damit man einen Kartenturm beliebiger Höhe bauen kann?
		\item[d)] Wir nehmen an, dass $\nu$ und $\mu$ so gewählt sind, dass man den Turm beliebig hoch bauen kann. Trotzdem zeigt die Erfahrung, dass es eine gewisse Grenze gibt, ab der der Turm instabill wird. Das liegt daran, dass die Karten ab einer bestimmten Kraft anfangen, sich zu biegen. Diese kritische Kraft ist gegeben durch
		\begin{equation}
		F_k = \frac{\pi^2 b d^3E }{12\ell^2},
		\end{equation}
		wobei $b$ die Breite, $\ell$ die Länge, $d$ die Dicke und $E$ das sog. Elastizitätsmodul, also eine Werkstoffkonstante, der Karte ist, siehe Abb. \ref{fig:buck}.\\
		Für Spielkarten gelten ungefähr folgende Größen: $\ell = 90~\mathrm{mm}$, $b=60~\mathrm{mm}$, $d = 0.3~\mathrm{mm}$, $E = 200~\mathrm{N\cdot mm^2}$. 
		Schätze mit diesen Angaben die maximale Anzahl der Stockwerke ab.
		
	\end{itemize}
	\begin{figure}[H]
		\centering
		
		\begin{subfigure}[b]{0.3\textwidth}
			\centering
			\includegraphics[scale = 0.1]{/Users/aaron/Desktop/problemset/ROLF-KoZi/kh.jpg}
			\caption{Ein Kartenhaus, \footnotesize (JMP, CC BY-SA 2.0 de)}
			\label{fig:kh}
		\end{subfigure}
		\hfill
		\begin{subfigure}[b]{0.2\textwidth}
			\centering
			\begin{tikzpicture}		
			\draw[dashed] (0,0)--(0.5,1)--(0.5,1)--(1,0);
			\draw[dashed] (1.1,0)--(1.6,1)-- (1.6,1)--(2.1,0);
			\draw[dashed] (0.5,1)--(1.6,1);
			\draw[dashed] (0.525,1)--(1.025,2)--(1.525,1);
			\draw[thick] (-0.5,0)--(2.6,0);
			\fill[pattern = north east lines] (-0.5,0) rectangle (2.6,-0.1);
			
			\end{tikzpicture}
			\caption{Grundzellen}
			\label{fig:cells}
		\end{subfigure}
		\hfill
		\begin{subfigure}[b]{0.2\textwidth}
			\centering
			\begin{tikzpicture}
			\tkzDefPoint(0,0){A}
			\tkzDefPoint(0.5,1){B}
			\tkzDefPoint(1,0){C};
			\tkzMarkAngle[scale = 0.7](C,A,B)
			\node at (0.24,0.17) {$\theta$};
			\draw[dashed] (0,0)--(0.5,1)--(1,0);
			\draw[thick] (-0.2,0)--(1.2,0);
			\fill[pattern = north east lines] (-0.2,0) rectangle (1.2,-0.1);
			\draw[->] (0.5,1.5)--(0.5,1);
			\node at (0.7,1.3) {$\vec{F}$};
			
			\end{tikzpicture}
			\caption{Eine Zelle}
			\label{fig:ocell}
		\end{subfigure}
		\hfill
		\begin{subfigure}[b]{0.2\textwidth}
			\centering
			\begin{tikzpicture}
			\draw[<->] (0,-1)--(0,1);
			\draw (1.2,-0.9)--(1.2,1.1);
			\draw[thick] (1.2,1.1)--(0.5,1);
			\draw[thick] (1.2,-0.9)--(0.5,-1);
			%\draw (0.5,1)--(0.5,-1);
			\tkzDefPoint(1.2,1.1){A}
			\tkzDefPoint(0.5,1){B}
			\tkzDefMidPoint(A,B)\tkzGetPoint{C}
			\tkzDefShiftPoint[C](0,0.4){D}
			\tkzDrawSegment[<-](C,D)
			\tkzDefPoint(1.2,-0.9){E}
			\tkzDefPoint(0.5,-1){F}
			\tkzDefMidPoint(E,F) \tkzGetPoint{G}
			\tkzDefShiftPoint[G](0,-0.4){H}
			\tkzDrawSegment[->](H,G)
			\tkzDefPoint (0.5,-1){X}
			\tkzDefPoint(1.2,-0.9){Y}
			\draw[<->] (0.5,-1.5)--(1.2,-1.4);
			\tkzDefMidPoint(X,Y)\tkzGetPoint{Z}
			\tkzDefShiftPoint[Z](0,-0.5){d}
			\tkzLabelPoint[below](d){$b$}
			\tkzDefMidPoint(C,D)\tkzGetPoint{f}
			\tkzLabelPoint[above right](f){$\vec{F}$}
			
			
			\draw[thick] plot[smooth,tension = 2] coordinates {(1.2,1.1) (1.1,0) (1.2,-0.9)} ;
			\draw[thick] plot[smooth, tension = 2] coordinates{(0.5,-1) (0.4,0) (0.5,1)};
			\node at (-0.25,0) {$\ell$};
			
			
			\end{tikzpicture}
			\caption{gebogene Karte}
			\label{fig:buck}
		\end{subfigure}
	\end{figure}
\end{Exercise}







\begin{Exercise}[label = ampel, title = Gute Ampel, difficulty = 3, origin = Alte IPhO-Aufgabe ]
	Ein Auto nähert sich auf einer Straße einer Ampel mit der Geschwindigkeit $v_0$.  Der Gleitreibungskoeffizient zwischen Straße und Auto beträgt $\mu$.\\ Die Ampel springt von Grün auf Gelb, wenn der Abstand Auto-Ampel gerade $s_0$ beträgt. Wie lang muss die Ampel gelb sein, damit der Autofahrer, unabhängig von $v_0$, entweder vor der Ampel zum Stehen kommt, oder aber mit konstanter Geschwindigkeit die Ampel kreuzt?
\end{Exercise}	
\begin{Answer}[ref = ampel]
	Diese Aufgabe kann man über mehrere Wege lösen.\\
	 Am Einfachsten geht es mit der Energieerhaltung. Wenn das Auto nach zurücklegen der Strecke $s_0$ auf jeden Fall zum Stehen gekommen sein, muss die auf dieser Strecke aufgebrachte Reibungsarbeit $E_r$ mindestens so groß sein, wie die kinetische Energie $E_{kin}$ vor dem Bremsen. Damit gillt
	 \begin{equation}\label{energycond}
	 E_{r} \geq E_{kin} \Rightarrow mg\mu s_0 \geq \frac{mv_0^2}{2}.
	 \end{equation}
	 Damit man vor der Ampel bei gegebener Strecke stehen bleiben kann, muss die Geschwindigkeit so gewählt werden, dass \eqref{energycond} erfüllt ist. Durch das Wurzelziehen ändert sich die Richtung der Ungleichung nicht, sodass man höchsten mit einer Geschwindigkeit von 
	 \begin{equation}\label{vfriccond}
	 	v_0\leq \sqrt{2s_0g\mu}
	 \end{equation}
	 vor der Ampel stehen bleiben kann.\\
	 Wenn man die Ampel bei gegebener Gelbphase ohne zu Bremsen kreuzen will, muss die Geschwindigkeit $v_0$ dafür groß genug sein, also
	 \begin{equation}\label{velcond}
	 	v_0 \geq  \frac{s_0}{t_g}
	 \end{equation}
	Da ein Bremsen aber immer möglich seien soll, können wir beide Bedingungen zusammen in einer Ungleichung schreiben
	\begin{equation}\label{stvcond}
		\sqrt{2s_0g\mu}\geq v_0 \geq \frac{s_0}{t}.
	\end{equation}
	Nach Aufgabenstellung soll die Zeit der Gelbphase $t_g$ so gewählt werden, dass die Bedingung \eqref{stvcond} unabhängig von der Geschwindigkeit ist. Das ist der Fall, wenn
	\begin{equation}\label{energytcond}
	\boxed{
	\sqrt{2s_0g\mu} \leq \frac{s_0}{t_g} \Rightarrow t_g \geq  \sqrt{\frac{s_0}{2g\mu}}}
	\end{equation}
	gilt.\\
	%Man kann die Aufgabe auch ohne Energieerhaltung nutzen. Dann muss man drei Bedingungen betrachten. Die erste Bedingung sagt, dass das Auto zum stehen gekommen ist, bevor die Ampel umschaltet, seine Geschwindigkeit also kleiner gleich null ist
	%\begin{equation}\label{traff:cond1}
%		v\left(t_g\right)\leq 0 \Rightarrow v_0 - g\mu t_g \leq 0 \Rightarrow v_0 \leq g \mu t_g.
%	\end{equation}
%	Gleichzeitig muss die Strecke, die das Auto in der Zeit $t_g$ bremsend zurücklegt, kleiner sein, als der Abstand vom der Ampel. Weil es sich um eine gleichmäßig beschleunigte Bewegung mit der Anfangsgeschwindigkeit $v_0$ handelt, muss also gelten
%	\begin{equation}\label{traff:cond2}
%	s\left(t_g\right) \leq s_0 \Rightarrow	-\frac{g\mu}{2}t_g^2+v_0t_g \leq s_0 \Rightarrow v_0 \leq \frac{s_0+\frac{g\mu}{2}t_g^2}{t_g}.
%	\end{equation}
%	Für den Fall, dass das Auto nicht bremst, sondern mit konstanter Geschwindigkeit weiterfährt, muss wieder gelten
%	\begin{equation}\label{traff:cond3}
%		v_0 \geq \frac{s_0}{t_g}.
%	\end{equation}
%	Wir können jetzt zuerst 
\end{Answer}





\end{document}
