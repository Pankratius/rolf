\begin{Exercise}[label = cmparallel, difficulty = 3, origin = P. Gnädig, title = Parallellandung]
	Wir betrachten eine Rakete die von der Erde gestartet wird, und deren Geschwindigkeitsvektor bei der Landung parallel zum Geschwindigkeitsvektor beim Start war. Der Start- und Landepunkt schließen einen Winkel $\vartheta$ mit dem Erdmittelpunkt ein. Fliegt die Rakete direkt über der Erde in einer Kreisbahn, dann benötigt sie dafür die Zeit $T_0$. \\
	Wie lange benötigt die Rakete für diesen besonderen Flug, und wie groß ist dabei die maximale Höhe, die sie erreichen kann?
\end{Exercise}