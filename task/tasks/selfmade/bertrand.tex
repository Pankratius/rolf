\begin{Exercise}[label = bertrand, origin = Aaron Wild, difficulty = 5, title = Betrands Theorem]
	Wir betrachten ein Potential der Form
	\begin{equation}
		U\left(r\right) = -\frac{k}{n},~n \neq 0, ~k \in \mathbb{R}
	\end{equation}
	Zeige, dass die einzigen Werte für $n$, für die ein Körper, der sich in diesem Potential bewegt, einen gebundenen und geschlossenen Orbit haben kann, $n = 1$ und $n=-2$ sind.\\
	Dafür kannst du benutzen, dass in einem solchen Feld der Orbit durch die Gleichung
	\begin{equation}\label{bertrand:binet}
		\frac{d^2u}{d\theta^2} + u =- \frac{m}{\ell^2} \cdot \frac{d}{du}\left(U\left(\nicefrac{1}{u}\right)\right)
	\end{equation}
	beschrieben wird, wobei $u:= \frac{1}{r}$ und $\theta$ der Drehwinkel ist.
\end{Exercise}