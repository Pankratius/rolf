\documentclass[a4paper]{article}
\usepackage{exercise}
%um nur aufgaben zu zeigen
%\usepackage[noanswer]{exercise} 
\usepackage{../images/preamble}
\usepackage{rotating}
\usetikzlibrary{decorations.pathmorphing}
\usetikzlibrary{decorations.markings}
\usetikzlibrary{arrows}
\usetikzlibrary{shapes.geometric}
\newcommand{\midarrow}{\tikz \draw[-triangle 90] (0,0) -- +(.02,0);}
\usepackage{xcolor}
%\usepackage{draftwatermark}
%\SetWatermarkText{\textsc{Draft 2}}
%\SetWatermarkScale{3}
%\SetWatermarkColor{red!30}

\usepackage[printwatermark]{xwatermark}
%\newsavebox\mybox
%\savebox\mybox{\tikz[color=red,opacity=0.3]\node{\textsc{Entwurf}};}
%\newwatermark*[
%allpages,
%angle=45,
%scale=10,
%xpos=-4cm,
%ypos=4cm
%]{\usebox\mybox}
\pagestyle{fancy}
\fancyhead[L]{\includegraphics[width=2cm]{../images/logo_scaled.pdf}}
\fancyhead[R]{\textsc{Lösung Aufgabenserie 4}}


\begin{document}
	\vspace*{-1cm}
	\parbox{4cm}{\vspace{-0.2cm}\includegraphics[width=5cm]{../images/logo_scaled.pdf}}
	\parbox{10.6cm}{\setstretch{2.0} \centering{ \huge \textsf{Aufgabenserie 4 
			}}\\
			Abgabe: 5. Oktober \\ \vspace*{-.5cm} }
		\vspace{0.5cm}

\thispagestyle{empty}
\begin{framed}
	\noindent
	\scriptsize
	Die Aufgaben sollten bis zum \textbf{5. Oktober} bearbeitet werden. Die Lösungen schickt ihr an \href{mailto:physikrolf@gmail.com}{physikrolf@gmail.com}.
	Jede Aufgabe hat eine bestimmte Anzahl an erreichbaren Punkten. Wie viele das sind, müsst ihr raten. Versucht, die Lösungen so genau wie möglich aufzuschreiben. Für besonders schnelle/gute/witzige Lösungen kann es Bonuspunkte geben.\\ Die aktuellen Aufgaben sowie alle alten Aufgabenserien mit Lösungen findet ihr auch auf \url{pankratius.github.io/rolf}. %\\\textit{Zu jeder Aufgabe gibt es jetzt Tipps. Die sollten beim Lösen der Aufgaben helfen.\\ Sollte das so sein macht bitte in euren Lösungen kenntlich, dass bestimmte Schritte von den Tipps und nicht von euch kommen. Darauf gibt es keinen Abzug, es ist nur für uns gut zu wissen.}
\end{framed}

\noindent

\begin{minipage}[b]{0.8\textwidth}
\begin{Exercise}[label = atwood, origin = Aaron Wild, title = Atwoodmaschinen, difficulty = 4]
	Eine Atwoodmaschine besteht im einfachsten Fall aus einer festen Rolle, an der (durch eine masselose Schnur verbunden) zwei Massen $m_1$ und $m_2$ hängen.
	\Question Bestimme für den Fall einer einfachen Atwoodmaschine die Beschleunigung des Systems sowie die Zugkraft im Seil. 
	\ExeText Betrachte jetzt eine unendlichen Atwoodmaschine, bei der alle Massen $m$ betragen.
	\Question Bestimme die Beschleunigung der obersten Masse, für den Fall, dass alle Massen gleichzeitig los gelassen werden.
\end{Exercise}
\end{minipage}
\hfill
\begin{minipage}[t]{0.2\textwidth}
	\centering
		\begin{tikzpicture}
		\draw[thick] (-1,0) -- (1,0);
		\fill[pattern = north east lines] (-1,0) rectangle (1,0.25);
		\draw[thick] (0,0)-- (0,-.5);
		\draw (0,-0.5) circle (0.25);
		\filldraw[black] (0,-0.5) circle (1pt);
		
		\draw (-.25,-0.5) -- (-.25,-1);
		\filldraw[black] (-.25,-1) circle (1.5pt);
		\node at (-.75,-1) {$m$};
		\draw (.25,-0.5)--(.25,-1);
		\draw (.25,-1) circle (0.25);
		\filldraw[black] (.25,-1) circle (1pt);
		\draw (0,-1) -- (0,-1.5);
		\filldraw[black] (0,-1.5) circle (1.5pt);
		\node at (-.5,-1.5) {$m$};
		\draw (.5,-1) -- (.5,-1.5);
		\draw (.5,-1.5) circle (0.25);
		\filldraw[black] (.5,-1.5) circle (1pt);
		\draw (.25,-1.5) -- (.25,-2);
		\filldraw[black] (.25,-2) circle (1.5pt);
		\node at (-.25,-2) {$m$};
		\draw (.75,-1.5) -- (.75,-2);
		\node at (.75,-2.25) {$...$};
		
		
		\end{tikzpicture}
\end{minipage}

\begin{Answer}[ref = atwood]
	\Question Weil die beiden Massestücken der einfachen Atwoodmaschine durch eine Seil fest verbunden sind, müssen sie sich beide mit der Beschleunigung $a$ bewegen, die eine Masse nach oben, und die andere nach unten.\\
	Wir definieren die Bewegungsrichtung der Masse $m_1$ (willkürlich!) als positiv, weswegen die der Masse $m_2$ negativ sein muss. Wenn die Spannkraft im Seil $F_s$ ist, gilt dann für die auf $m_1$ und $m_2$ wirkenden Kräfte
	\begin{subequations}\label{atwood:forces}
		\begin{equation}
			F_s - m_1 g = F_1
		\end{equation}
		\begin{equation}
			m_2 g - F_s = F_2.
		\end{equation}
	\end{subequations}
	In beiden Teilen von \eqref{atwood:forces} gilt nun, dass die Kraft gleich der Masse mal der wirkenden Beschleunigung ist, also $F_1 = m_1 a$ bzw. $F_2 = m_2 a$, sodass \eqref{atwood:forces} ein Gleichungssystem in den beiden Unbekannten $a$ und $F_s$ darstellt, welches einfach gelöst werden kann, um auf
	\begin{subequations}
			\begin{equation}\label{atwood:acc}
			\boxed{	a = \frac{\left(m_2-m_1\right)g}{m_1+m_2}}
			\end{equation}
			\begin{equation}
			\boxed{				T = \frac{2m_1m_2g}{m_1+m_2}.}
			\end{equation}
	\end{subequations}
	zu kommen. \\
	Gleichung \eqref{atwood:acc} macht auch intuitiv Sinn, weil hier für $m_1 = m_2$ für die Beschleunigung $a= 0$ folgt, was heißt, dass sich die Atwoodmaschine bei gleichen Massen nicht aus der stabilen Gleichgewichtslage bewegt (wie wir erwarten würden).
	\Question Dieser Aufgabenteil lässt sich durch eine Skalierungsargumentation für die ersten zwei Rollen lösen. Dazu betrachten wir die Spannung im ersten und zweiten Seil, $F_{S,1}$ bzw. $F_{S,2}$. Die Seilspannung muss sich zwischen dem ersten Massenstück und dem zweiten Seil aufgeteilt haben, sodass $F_{S,2} = \nicefrac{1}{2} F_{S,1}$ gelten muss. \\
	Gleichzeitig können wir analysieren, wie die Wirkung der Gravitations skalieren muss. Dazu überlegen wir uns zuerst, was passieren würden, wenn die Gravitationskraft der Erde um einen Faktor $\nu$ skaliert werden würde. Selbstverständlich würde dann auch die Gravitation um diesen Faktor $\nu$ skaliert werden. Dementsprechend würde aber auch die Spannkraft $F_s$ um einen Faktor $\nu$ skaliert werden, sodass die neue Spannkraft $\nu F_s$ beträgt.\\
	Wenn wir nun die zweite Rolle betrachten, und ihre Beschleunigung $a_2$ nennen, ist die effektiv wirkende Beschleunigung $a_2-g$, weil wir ja noch die Kraft durch die oberste Rolle betrachte müssen.\\
	Weil das \glqq Netzwerk\grqq{} aus Rollen aber unendlich groß sein soll, darf es keinen Unterschied geben, ob man nun die erste oder die zweite Rolle betrachtet. Folglich muss
	\begin{equation}\label{atwood:accsol1}
		\frac{T}{g} = \frac{\nicefrac{T}{2}}{g-a_2}
	\end{equation}
	gelten. Wenn wir jetzt nach $a_2$ umstellen, finden wir, dass $a_2 = \nicefrac{g}{2}$ gelten muss. Weil aber die Beschleunigung der zweiten Masse die gleiche sein muss wie die der ersten Masse, da die beiden ja über ein Seil miteinander verbunden sind, ist auch die Beschleunigung der obersten Masse $\nicefrac{g}{2}$.\\
	%Für den zweiten Lösungsansatz vergleichen wir zuerst den Fall einer Masse $m$, die ohne jegliche Rolle von der Wand hängt, und wieder zwei Massen $m_1$ bzw. $m_2$, die über eine feste Rolle verbunden sind. Die Frage ist nun, wie $m$ gewählt werden muss, damit die Spannkraft $F_{s,m}$ so groß ist, wie die, die in dem System aus $m_1$ und $m_2$ wirkt. 
	%Dazu berechnen wir zuerst wieder die Beschleunigungen bedingt durch Spannkraft und Gravitationskraft. Für den Fall der Ersatzmasse $m$ ist das relativ einfach, und es gilt einfach
	%\begin{equation}\label{key}
	%	
	%\end{equation}
	
\end{Answer}
 
\begin{Exercise}[difficulty = 3, origin = 3. Runde IPhO-Auswahlwettbewerb 2008, title = Bimetallstreifen, label = bms ]
	\hspace{4pt}
	Ein geklebter Bimetallstreifen besteht aus zwei Metallschichten der Dicke $d$, die  Wärmeausdehungskoeffizienten $\alpha_1$ bzw. $\alpha_2$ ($\alpha_2>\alpha_1$) haben. Im Anfgangszustand ist der Streifen gerade. 
	Wie groß ist der Krümmungsradius, wenn der Streifen um $\Delta T$ erwärmt wird? Was passiert im Grenzfall fast gleicher Wärmeausdehungskoeffizienten ($\alpha_2\rightarrow\alpha_1$)?
\end{Exercise}
\newpage
\begin{Exercise}[label = trinity, origin = Aaron Wild, title = {Project Trinity}, difficulty = 3]
	Die Ausbreitung einer halbkreisförmigen Schockwelle hängt von der Energie $E$ der Explosion sowie der Dichte der Luft $\rho$ ab.
	\Question Bestimme eine Gleichung, die den Radius $R$ der Schockwelle als Funktion der Zeit $t$ nach der Explosion angibt.
	\ExeText Die folgenden Bilder zeigen die Schockwelle nach dem ersten Atombombentest der USA 1945, \textit{Project Trinity}:
		\begin{figure}[h]
		\begin{subfigure}[b]{0.5\textwidth}
			\centering
			\includegraphics[scale = 0.25]{../tasks/selfmade/trinityim.png}
		\end{subfigure}
		\begin{subfigure}[b]{0.5\textwidth}
			\centering
			\includegraphics[scale = 0.25]{../tasks/selfmade/trinityim2.png}
		\end{subfigure}
		\caption{Massstabsgerechte Abbildung einer Schockwelle nach der Detonation}
		\label{fig:trinityim}
		\end{figure}
	\Question Nutze diese Bilder, um das Ergebnis aus 1. zu bestätigen.
\end{Exercise}



\end{document}