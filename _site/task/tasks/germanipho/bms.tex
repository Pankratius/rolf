\begin{Exercise}[difficulty = 3, origin = 3. Runde IPhO-Auswahlwettbewerb 2008, title = Bimetallstreifen, label = bms ]
	\hspace{4pt}
	Ein geklebter Bimetallstreifen besteht aus zwei Metallschichten der Dicke $d$, die  Wärmeausdehungskoeffizienten $\alpha_1$ bzw. $\alpha_2$ ($\alpha_2>\alpha_1$) haben. Im Anfgangszustand ist der Streifen gerade. 
	Wie groß ist der Krümmungsradius, wenn der Streifen um $\Delta T$ erwärmt wird? Was passiert im Grenzfall fast gleicher Wärmeausdehungskoeffizienten ($\alpha_2\rightarrow\alpha_1$)?
\end{Exercise}
\begin{Answer}[ref = bms]
	\begin{figure}[h]
		\centering
			\begin{tikzpicture}[scale =1.5]
			\clip(-0.47848681813981486,-1.0992522773586708) rectangle (2.556743060771764,1.7
			191547261822666);
			\draw [shift={(0.,0.)},thick] plot[domain=0.:0.9394498790362418,variable=\t]({1.*1.5*cos(\t r)+0.*1.5*sin(\t r)},{0.*1.5*cos(\t r)+1.*1.5*sin(\t r)});
			\draw [shift={(0.,0.)},thick] plot[domain=0.:0.9394498790362418,variable=\t]({1.*1.*cos(\t r)+0.*1.*sin(\t r)},{0.*1.*cos(\t r)+1.*1.*sin(\t r)});
			\draw [shift={(0.,0.)}] plot[domain=0.:0.9394498790362418,variable=\t]({1.*1.25*cos(\t r)+0.*1.25*sin(\t r)},{0.*1.25*cos(\t r)+1.*1.25*sin(\t r)});
			\draw[thick] (0.5853462909316203,0.800551302621677)-- (0.8851425892165119,1.2105689636734724);
			\draw[thick] (1.,0.)-- (1.5,0.);
				
			\tkzDefPoint(0,0){O}
			\tkzDefPoint(1,0){A}
			\tkzDefPoint(53.83:1){B}
			\tkzMarkAngle[scale = .5](A,O,B)
			\tkzLabelAngle[pos = .55](A,O,B){$\varphi$}
			\tkzDrawPoints(O)
			\tkzDrawSegment[dashed](A,O)
			\tkzDrawSegment[dashed](B,O)
			
			\draw[|-|] (0,-.5) --  (1.25,-.5) node [midway, below] {$R$};
			
			\draw[|-|] (1.25,-0.25) --(1.5,-0.25) node [midway, above] {$d$};
			\end{tikzpicture}
	\end{figure}
	Wir bezeichnen die ursrpüngliche Länge des Bimetallstreifens mit $\ell$. Durch das Erwärmen erhalten wir zwei neue Längen, die gegeben sind durch
	\begin{equation}\label{bms:therm}
		\ell_1 = \ell\left(1+\alpha_1\Delta T\right) ~\mathrm{und}~\ell_2 = \ell\left(1+\alpha_2 \Delta T\right).
	\end{equation}
	Diese Längenänderung führt zu einer kreisförmigen Krümmung des Streifens mit eben dem Krümmungsradius $R$ an der Verbindungsstelle. Den Winkel dieser Krümmung nennen wir $\varphi$. \\
	Damit können wir die neuen Längen jetzt auch als Bogenlängen über dem Winkel $\alpha$ darstellen:
	\begin{equation}\label{bms:arc}
		\ell_1 = \left(R-\frac{d}{2}\right) \varphi ~\mathrm{und}~\ell_2 = \left(R+\frac{d}{2}\right)\varphi.
	\end{equation}
	Die beiden Gleichungen \eqref{bms:therm} und \eqref{bms:arc} stellen nun ein Gleichungssystem da, was man nach $R$ umstellen kann. Wenn man das macht, kommt man auf
	\begin{equation}\label{bms:rsol}
		\boxed{
			R = d\cdot \frac{2+\left(\alpha_1 + \alpha_2\right) \Delta T}{2\left(\alpha_2 - \alpha_1\right)\Delta T}
			}
	\end{equation}
	Im Grenzfall fast gleicher Längenausdehnungskoeffizienten sollte es nicht zu einer Krümmung kommen, weil sich ja beide Streifen gleich stark ausdehnen. Dementsprechend sollte der Krümmungsradius gegen unendlich gehen.\\
	Das sehen wir auch an dem dimensionslosen, zweiten Faktor in Gleichung \eqref{bms:rsol}. Im Fall $\alpha_2\rightarrow \alpha_1$ kommt der Nenner des Bruchs immer näher an null, sodass der Krümmungsradius ohne jede Grenze steigt, also keine Krümmung vorliegt.
\end{Answer}