\begin{minipage}[b]{0.8\textwidth}
\begin{Exercise}[label=ferm1,title = Linsenform, difficulty = 3, origin = Aaron Wild]
	Wir betrachten Licht, welches aus dem unendlichen im Vakuum auf ein gekrümmtes Material mit Brechungsindex, welches das Licht im Punkt $F$ fokussiert. Bestimme, welche Form die Oberfläche haben muss (also die Funktion $y\left(x\right)$), und ab welcher Höhe $y_{max}$ es nicht mehr möglich ist, dass das Licht fokussiert wird.
\end{Exercise}
\end{minipage}
\begin{minipage}[b]{0.2\textwidth}
	\centering
\begin{tikzpicture}
	\draw[->,dotted](0,0) -- (0,1.7);
	\draw[->,dotted](0,0) -- (1.75,0);
	\draw[scale=1,domain=0:1,smooth,variable=\x,blue] plot ({\x},{1.5*\x^(0.5)});
	\begin{scope}[ thick,decoration={
		markings,
		mark=at position 0.5 with {\arrow{>}}}
	] 
	\draw[thick] (-0.5,0) -- (-0.1,0);
	\draw[thick] (-0.5,0.75) --(0.25,0.75);
	\draw[thick](-0.5,1) -- (0.444444,1);
	\draw[thick] (-0.5,0.5) -- (0.111111,0.5);
	\draw[thick] (-0.5,1.25) -- (0.694444,1.25);
	\draw[postaction={decorate}] (0.25,0.75) -- (1.5,0);
	\draw[postaction={decorate}] (0.4444,1) -- (1.5,0);
	\draw[postaction={decorate}] (0.111111,0.5)--(1.5,0);
	\draw[postaction={decorate}](0.694444,1.25)--(1.5,0);
	\end{scope}
	\node at (-0.2,1.78) {\small $y$};
	\node at (1.95,0){\small$x$};
	\node at (1.7,0.2){\small $F$};
	\filldraw[black] (1.5,0)circle(1.5pt);
	\draw[<->] (0,-0.5) --  (1.5,-0.5)  node[midway, fill = white]{$f$};
	\node at (1.2,1.2) {$n$};
	
\end{tikzpicture}
\end{minipage}
\begin{Answer}[ref = ferm1]
	Nach dem Fermatschen Prinzip müssen alle Lichtwege, die zum Brennpunkt führen, die gleiche Zeit benötigen. Dafür reicht es zu wissen, dass die Ausbreitungsgeschwindigkeit in einem Medium mit der Brechzahl $n$ gleich $v = \nicefrac{c}{n}$ ist, wobei $n$ die Lichtgeschwindigkeit im Vakuum ist. Die Zeit, die der Lichtstrahl für eine Strecke $s$ braucht, ist dann einfach $\frac{s}{v}= \frac{ns}{c}$. Weil unabhängig von der Brechzahl jeder Term für die Strecke also den Faktor $\frac{1}{c}$ hat, brauchen wir den nicht mitnehmen, und können stattdesen Strecken untersuchen, die einfach mit der entsprechenden Brechzahl $n$ multipliziert werden. Man spricht dann von der \textit{optischen Weglänge}. Es müssen also die optischen Weglängen aller Strahlen gleich sein.\\
	Um die vergleichen zu können, reicht es aus, ab der $y$-Achse zu messen, weil bis dahin alle Strahlen den gleichen Weg zurückgelegt haben (weil sie parallel sind).\\
	Am einfachsten lässt sich die optische Weglänge für den Strahl ausrechnen, der entlang der $x$-Achse verläuft. Sie ist einfach 
	\begin{equation}\label{ferm1:triv}
		s_1 = n f,
	\end{equation}
	weil der senkrecht einfallende Strahl nicht gebrochen wird.\\
	Anders sieht es mit dem Strahl aus, der im Punkt $\left(x,y\right)$ auf die Oberfläche trifft, (siehe Skizze). Hier gilt für die Brechung das Brechungsgesetz
	\begin{equation*}
		\sin \alpha = n \sin \beta,
	\end{equation*}
	wobei $\alpha$ der Einfallswinkel (relativ zum Lot) ist und $\beta$ der Ausfallswinkel.\\
	Wie man in der Skizze sieht, setzt sich der Lichtweg dieses Strahls aus zwei Teilen zusammen. Der eine in Luft beträgt einfach $s_{2,1} = x$. Der andere ist die Hypothenuse im Dreieck, welches mit dem Brennpunkt $F$ gebildet werden kann. Dort ist der Lichtweg
	\begin{equation}
		s_{2,2} = n \sqrt{y^2+\left(f-x\right)^2}.
	\end{equation}
	Der gesamte Lichtweg dieses Strahls ist also 
	\begin{equation}\label{ferm1:stot}
		s_{2} = s_{2,1} + s_{2,2} = x + n\sqrt{y^2+\left(f-x\right)^2}.
	\end{equation}
	Gleichsetzten von \eqref{ferm1:triv} und \eqref{ferm1:stot} führt auf
	\begin{equation}
		n f = x + n \sqrt{y^2+\left(f-x\right)^2} \Rightarrow \frac{n f- x}{n} =  \sqrt{y^2+\left(f-x\right)^2} \Rightarrow y = \sqrt{ \left( \frac{n f- x}{n} \right)^2 - \left(f-x\right)^2}
	\end{equation}


	

\end{Answer}

	
