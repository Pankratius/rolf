\documentclass[a4paper]{article}

%um nur aufgaben zu zeigen

\renewcommand{\title}{Aufgabenseminar\\ Klassische Mechanik}
\newcommand{\titleh}{Aufgabenseminar Klassische Mechanik}



\usepackage[noanswer]{exercise} 
\usepackage{../images/preamble}
\usepackage{rotating}
\usetikzlibrary{decorations.pathmorphing}
\usetikzlibrary{decorations.markings}
\usetikzlibrary{arrows}
\usetikzlibrary{shapes.geometric}
\usepackage{mathrsfs}
\newcommand{\midarrow}{\tikz \draw[-triangle 90] (0,0) -- +(.02,0);}
\usepackage{xcolor}
%\usepackage{draftwatermark}
%\SetWatermarkText{\textsc{Entwurf}}
%\SetWatermarkScale{6}
%\SetWatermarkColor{red!30}

\pagestyle{fancy}
\fancyhead[L]{\includegraphics[width=2cm]{../images/logo_scaled.pdf}}
\fancyhead[R]{\textsc{\titleh}}


\renewcommand{\ExerciseHeader}{\textsf{\textbf{\ExerciseTitle} (\ExerciseOrigin)}\smallskip\newline}
\renewcommand{\AtBeginExercise}{\hspace{-0.66em}}


\begin{document}
	\vspace*{-1cm}
	\parbox{4cm}{\vspace{-0.2cm}\includegraphics[width=5cm]{../images/logo_scaled.pdf}}
	\parbox{10.6cm}{\setstretch{2.0} \centering{ \huge \textsf{\title
			}}\\\url{pankratius.github.io/rolf}
			 }
		\vspace{0.5cm}
	
	

\thispagestyle{empty}


\noindent

\begin{Exercise}[label = adi, title = Pendel, difficulty = 5, origin = Aaron Wild]
	Wir betrachten ein harmonisches Pendel, desen Länge sich langsam ändert. Wie ändert sich die Amplitude der Schwingung?
\end{Exercise}
\begin{Exercise}[label = grassh, title = fauler Grasshüpfer, origin = P.Gnädig, difficulty = 5]
	Ein fauler Grasshüpfer möchte über einen Baumstumpf mit Radius $r = 20~\mathrm{cm}$ springen.\\
	Wie groß ist die dafür mindestens benötigte Geschwindigkeit, wenn der Luftwiderstand vernachlässigt werden kann?
\end{Exercise}
\begin{Exercise}[origin = Jaan Kalda, title = Fuchsjagd, difficulty = 4, label = fox]
	Ein Hund jagt einen Fuchs, welcher sich mit einer konstanten Geschwindigkeit $v$ entlang einer Geraden bewegt. Der Hund bewegt sich ebenfalls mit $v$, jedoch ist sein Geschwindigkeitsvektor $\mathbf{v}$ immer nach dem Fuchs ausgerichtet. Am Anfang befindet sich der Hund senkrecht zu dem Fuchs, und die beiden haben einen Abstand von $\ell$. Was ist der minimale Abstand zwischen den beiden während der Verfolgung?
\end{Exercise}
\begin{Exercise}[origin = Jaan Kalda, difficulty = 4, title = Keil, label = wedge1]
	Wir betrachten eine kleine Kugel der Masse $m$, welche auf einem Keil der Masse $M$ mit Neigungswinkel $\beta$ liegt. Dabei ist die Kugel mit einem masselosen Faden und einer (ebenfalls masselosen Rolle) an der Wand befestigt. Finde die Geschwindigkeit des Keils.
\end{Exercise}

\begin{Exercise}[title = Flugzeuge, origin = Jaan Kalda]
	Zwei Flugzeuge fliegen auf gleicher Höhe mit den Geschwindigkeiten $v_1 = 600~\mathrm{\frac{km}{h}}$ und $v_2 = 800~\mathrm{\frac{km}{h}}$. Dabei befinden sie sich am Anfang auf den Eckpunkten eines gleichschenklig-rechtwinkligen Dreiecks mit der Seitenlänge $a = 20~\mathrm{km}$. Berechne den geringsten Abstand zwischen den beiden Flugzeugen unter der Annahme konstanter Geschwindigkeit.
\end{Exercise}

\begin{Exercise}[title = Wassertropfen, origin = Lukas Rettenmeier]
	Eine Wolke besteht aus einer Ansammlung von sehr kleinen Wassertröpfechen, die homogen im Raum verteilt sind. Nun fällt ein großer Wassertropfen durch die Wolke, wobei der Tropfen das Wasser jedes Tröpfchens, mit dem er zusammenstoßt, in sich aufnimmt. Nach einer langen Zeit wird sich der Tropfen mit einer konstanten Beschleunigung bewegen. Wie groß ist diese?
\end{Exercise}

\end{document}
