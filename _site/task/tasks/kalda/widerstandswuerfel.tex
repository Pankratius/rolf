\begin{Exercise}[label = wdsw, title = Widerstandswürfel , difficulty = 2, origin ={Jaan Kalda,\url{http://www.ioc.ee/~kalda/ipho/}}]
	Gegeben ist ein Würfel, wobei jede der Kanten einen Widerstand von $R = 1~\mathrm{\Omega}$ hat.\\
	Wie groß ist der Widerstand entlang einer Raumdiagonale?
\end{Exercise}
\begin{Answer}[ref=wdsw]
	Wir wollen den Widerstand zwischen den Punkten $X$ und $Y$ bestimmen, also entlang der Raumdiagonale (siehe Abb. \ref{fig:wdsws1}). Weil die Raumdiagonale eine Symmetrieachse ist, sollte das Problem symmetrisch sein, und deswegen eine recht einfache Lösung haben.\\
	Um die zu finden, schauen wir uns die Punkte $a$ und $b$ an, die jeweils den gleichen Abstand von $X$ und $Y$ haben. Diese müssen jeweils auf dem gleichen Potential liegen. Um zu sehen, warum das so ist, stellen wir uns vor, dass der Strom $I$ aus dem Punkt $X$ in die drei Punkte $a$ fließt. Weil alle drei Punkte gleich weit von $X$ entfernt sind (bzw. zwischen jedem Punkt $a$ und $X$ jeweils der Widerstand $R$ ist), muss sich der Strom gleichmäßig aufteilen. Gleichzeitig ist aber jeder Punkt $a$ mit zwei Punkten $b$ verbunden ist, muss sich auch hier der Strom gleichmäßig aufteilen. Also müssen alle Punkte $a$ auf dem gleichen Potential liegen, und auch alle Punkte $b$ auf einem anderen selben Potential.\\
	Damit können wir einen neuen Stromkreis zeichnen, in dem es nur einen Punkt $a$ und nur einen Punkt $b$ gibt, siehe Abb. \ref{fig:wdsws2}. Dieser ist eine einfache Schaltung von Widerständen, die alle den Wert $R$ haben. Werten wir die einzelnen Parallelschaltungen aus, so trägt die ersten einen Wert von $\frac{R}{3}$ bei, und die zweite einen Wert von $\frac{R}{6}$ und die dritte wieder $\frac{R}{3}$. Der Gesamtwiderstand ergibt sich aus der Reihenschaltung der drei, also $R_g = 2\cdot \frac{R}{3}+ \frac{R}{6} = \frac{5R}{6}$.
	

\end{Answer}
\begin{figure}[h]
\begin{subfigure}[b]{0.4\textwidth}
	\centering
	
	\tdplotsetmaincoords{80}{120}
\begin{tikzpicture}[scale=3, tdplot_main_coords,axis/.style={->},thick]  

    \coordinate (O) at (0,0,0);
    \tdplotsetcoord{P}{1.414213}{54.68636}{45}
    
    \draw (O) -- (Py) -- (Pyz) -- (Pz) -- cycle;
    \draw(O) -- (Px) -- (Pxy) -- (Py) -- cycle;
    \draw (O) -- (Px) -- (Pxz) -- (Pz) -- cycle;
    \draw(Pz) -- (Pyz) -- (P) -- (Pxz) -- cycle;
    \draw (Px) -- (Pxy) -- (P) -- (Pxz) -- cycle;
    \draw (Py) -- (Pxy) -- (P) -- (Pyz) -- cycle;
    \node at (0,-0.1,0.05) {$Y$};
    \node at (1,1,0.8) {$X$};
    \node at (-0.15,-0.1,0.875){$b$};
    \node at (1,0,0.85){$a$};
    \node at (-0.1,0.8,0.87){$a$};
    \node at (1,1,-0.01){$a$};
    \node at (0,0.9,0){$b$};
    \node at (1,0.05,-0.01){$b$};
	\foreach \x in {0,0.82}
	\foreach \y in {0,0.82}
	\foreach \z in {0,0.82}
	 {\draw[fill=black] (\x,\y,\z) circle (0.5pt);}    

\end{tikzpicture}

	\caption{Skizze des Wurfels}
	\label{fig:wdsws1}
\end{subfigure}
\begin{subfigure}[b]{0.6\textwidth}
	\centering
	\begin{circuitikz} \draw
	(0,0) to (0.5,0)
	(0.5,-0.5) to (0.5,0.5)
	\foreach \h in {-0.5,0,0.5}
		{(0.5,\h) to[R] (2.5,\h)}
	(2.5,0.5) to (2.5,-0.5)
	(2.5,0) to (3,0)	
 	(3,-1.5) to (3,1.5)
	\foreach \h in {-1.5,-1,-0.5,0.5,1,1.5}
	{(3,\h) to[R] (5,\h)}
	(5,1.5) to (5,-1.5)
	(5,0) to (6,0)
	(6,-0.5) to (6,0.5)
	\foreach \h in {-0.5,0,0.5}
	{(6,\h) to[R] (8,\h)}
	(8,0.5) to (8,-.5)
	(8,0) to (8.5,0)
	;
	\node at (-0.2,0) {$X$};
	\filldraw[black] (0,0) circle (1.5pt);
	\node at (8.7,0) {$Y$};
	\filldraw[black] (8.5,0) circle (1.5pt); 
	\node at (2.75,0.3) {$a$};
	\filldraw[black] (2.75,0) circle (1.5pt);
	\node at (5.5,0.3) {$b$};
	\filldraw[black] (5.5,0) circle (1.5pt);
\end{circuitikz}
	\caption{vereinfachtes Schaltbild}
	\label{fig:wdsws2}
\end{subfigure}
\caption{Schaltskizzen zum Aufbau}
\end{figure}
