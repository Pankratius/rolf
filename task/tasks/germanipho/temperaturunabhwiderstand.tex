\begin{Exercise}[label = tempind, title = Temperaturunabhängiger Widerstand, difficulty = 2, origin = Alte IPhO-Aufgabe ]
Wir betrachten zwei Materialien $A$ und $B$, mit spezifischem Widerstand $\rho_A$ und $\rho_B$, die beide die gleiche Fläche haben. Die Temperaturkoeffizienten der beiden Widerstände sind $\alpha_A$ und $\alpha_B$, und geben an, um wieviel sich der Widerstand bei einer Temperaturänderung $\Delta T$ ändert. Es ist $\alpha_A < 0$ und $\alpha_B > 0$. \\
In welchem Längenverhältnis müssen die Längen der beiden Widerstände $\ell_A$ und $\ell_B$ stehen, damit der Gesamtwiderstand einer Reihenschaltung der beiden bei einer Temperaturänderung $\Delta T$ konstant bleibt?
\end{Exercise}
\begin{Answer}[ref=tempind]
Wenn die Temperaturänderung $\Delta T$ nicht zu groß ist, kann man den neuen Widerstand $R$ nach dieser Temperaturänderung aus dem alten Widerstand $R_0$ ausrechnen mit
\begin{equation}\label{tempunwds:rchange}
	R\left(\Delta T\right) = R_0 + \alpha \Delta T R_0 = R_0\left(1+\alpha \Delta T\right),
\end{equation}
wobei $\alpha$ der Temperaturkoeffizient ist.\\
Gleichzeit ist der Widerstand $R$ eines Körpers der Länge $\ell$, der Fläche $A$ und dem spezifischen Widerstand $\rho$ 
\begin{equation}\label{tempunwds:rdef}
	R = \frac{\rho \ell}{A}.
\end{equation}
Die zwei gegebenen Widerstände $R_A$ und $R_B$ sind in Reihe geschaltet, also ist der Gesamtwiderstand $R_g$ 
\begin{equation}\label{tempunwds:rg}
	R_g = R_A + R_B =\frac{1}{A}\left(\rho_A \ell_A + \rho_B \ell_B\right) 
\end{equation}
Wenn wir annehmen, dass die Temperaturänderung so klein ist, dass sich die Länge der Widerstände nicht ändert, können wir \eqref{tempunwds:rchange} genauso führ den spezifischen Widerstand schreiben
\begin{equation}\label{tempunwds:rhochange}
	\rho\left(\Delta T\right) = \rho_0 + \alpha \Delta T \rho_0 = \rho_0\left(1+\alpha \Delta T\right).
\end{equation}
Schreiben wir diesen Ausdruck nun für $R_A$ und $R_B$, so erhalten wir als Gesamtwiderstand in Abhängigkeit von der Temperatur aus \eqref{tempunwds:rg}
\begin{equation*}
	\tilde{R}_g\left(\Delta T\right) = \frac{1}{A}\left[\left(\rho_A + \rho_A \alpha_A \Delta T\right)\ell_A + \left(\rho_B + \rho_B \alpha_B \Delta T\right) \ell_B \right].
\end{equation*}
Ausklammern Terme führt auf
\begin{equation}\label{tempunwds:rgtemp}
	\tilde{R}_g\left(\Delta T\right) =\underbrace{\frac{1}{A}\left[ \rho_A \ell_a + \rho_B \ell_B\right]}_{= R_g~  \mathrm{aus}~ \eqref{tempunwds:rg}} +\frac{\Delta T}{A} \left(\rho_A \alpha_A \ell_A + \rho_B \alpha_B \ell_B \right).
\end{equation}
Der erste Term ist genau der Gesamtwiderstand $R_g$ aus \eqref{tempunwds:rg} vor der Temperaturänderung. Der zweite Term ist ein Produkt aus $\Delta T$ und einem anderen Faktor. Damit der Gesamtwiderstand $\tilde{R}_g$ also unabhängig von der Temperaturänderung $\Delta T$ immer genau $R_g$ ist, muss der Faktor nach $\Delta T$ null sein, da dieser ganze Term dann keinen Einfluss hat. Also ist
\begin{equation}
\boxed{
	\rho_A \alpha_A\ell_A + \rho_B \alpha_B\ell_B = 0 \Rightarrow \frac{\ell_A}{\ell_B} = -\frac{\rho_B\alpha_B}{\rho_A\alpha_A}}.
\end{equation}
Das Minuszeichen muss da sein, da $\alpha_A$ auch negativ sein soll, und das Längenverhältnis nicht negativ sein sollte.
\end{Answer}