\section{Lenz-Runge-Vektor}
Es gibt keinen wirklich schönen Weg, die geometrische Form der Bahn zu bestimmen - man muss eigentlich immer ein doofes Interal ausrechnen. Zum Glück fällt aber irgendwie eine weitere Erhaltungsgröße vom Himmel, der \emph{Lenz-Runge-Vektor} $\vec{A}$:
\[
  \vec{A} = m\dot{\vec{x}}\times \vec{L} - k\frac{\vec{x}}{|\vec{x}|}.
\]
Nachzurechnen, dass $\vec{A}$ für Lösungen von \eqref{eom0} konstant ist, dauert ein wenig.
 %\begin{align*}
  %&\frac{\partial \vec{A}}{\partial t} &= m\frac{\partial}{\partial t} \left(\dot{\vec{x}}\times %\vec{L}\right) - k\frac{\partial}{\partial t}\left(\frac{\vec{x}}{|\vec{x}|}\right)  % \frac{\partial}{\partial t} \left(\dot{\vec{x}}\times \vec{L}\right) \\&&= m\ddot{\vec{x}}\times \vec{L},
% \end{align*}
%weil $\dot{\vec{L}}=0$ ist.
