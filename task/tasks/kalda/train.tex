\begin{Exercise}[title = Tunnel, origin = {EstPhO 2009}, difficulty = 3, label = train]
	Ein Zug fährt mit einer Geschwindigkeit $v$ und einer Leistung $P$ durch einen langen zylindrischen Tunnel, dessen halbkreisförmige Öffnung einen Durchmesser $d$ hat.\\
	Die Anfangstemperatur im Tunnel beträgt $T_0$, der Luftdruck ist $p_0$, die molare Masse von Luft ist $M$. Während der Durchfahrt bleibt der Druck im Tunnel näherungsweise konstant. Dann ist die molare Wärmekapazität von Luft durch $C_p$ gegeben.\\
	Welche Temperatur hat die Luft im Tunnel nach der Zugdurchfahrt?\\
	\textit{Hinweis: Wie schon auf dem letzten Blatt kann die Luft auch in dieser Aufgabe als \textit{ideales Gas} angenommen werden. Das heißt insbesondere auch, dass die Gleichung $pV = NRT$ für die Luft erfüllt ist. \\
	Dabei ist $p$ der Gasdruck, $V$ das Volumen, $N$ die Stoffmenge und $T$ die Temperatur der Luft. $R$ ist die sog. idelle Gaskonstante. Für zweiatomige Gase wie Luft gilt in guter Näherung $C_P = \nicefrac{7}{2}\cdot R$.}
\end{Exercise}