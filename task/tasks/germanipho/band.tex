\begin{Exercise}[title = Lichtband, label = band, difficulty = 3, origin = {1. Runde im Auswahlwettbewerb zur IPhO 2006}]
Eine Lichtquelle in Form eines dünnen Bandes der Länge $\ell = 10~\mathrm{cm}$ liegt auf der optischen Achse einer dünnen Sammellinse der Brennweite $f = 5~\mathrm{cm}$ und dem Durchmesser $d = 1~\mathrm{cm}$. Der minimale Abstand der Lichtquelle zur Sammellinse beträgt $10~\mathrm{cm}$. Wie groß ist der minimale Durchmesser des entstehenden Lichtflecks, den die Lichtquelle auf einem senkrecht zur optischen Achse stehenden, verschiebbaren Schirm erzeugt?
\end{Exercise}