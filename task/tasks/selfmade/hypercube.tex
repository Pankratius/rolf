\begin{Exercise}[label = hypercube, origin = {Max Marienhagen, Aaron Wild}, difficulty = 5, title =  Widerstandswürfel]
Ein $n$-dimensionaler Hyperwürfel ist die Verallgemeinerung eines Würfels auf $n$ Dimensionen. Seine Konstruktion kann man sich so vorstellen, das ein $n-1$-dimensionaler Hyperwürfel im $n$-dimensionalen Raum parallelverschoben wird, und man das daraus entstandene Volumen betrachtet.\\
Ein solcher $n$-dimensionaler Hyperwürfel hat $2^n$ Eckpunkte und $n2^{n-1}$ Seitenkanten.\\
Wir betrachten nun einen $n$-dimensionalen Hyperwürfel ($n\geq 1$), bei dem alle Seitenkanten einen Widerstand von $r$ haben. Zeige, dass der Widerstand zwischen zwei benachbarten Eckpunkten 
\begin{equation}
	R = \frac{2-2^{1-n}}{n}r = \frac{2^n-1}{n2^{n-1}}r
\end{equation}
beträgt. Überlege dir an einem $n$ deiner Wahl, dass das Ergebnis dort sinnvoll ist.
\end{Exercise}
\begin{Answer}[ref = hypercube]
	Diese Aufgabe kann durch ein Superpositionsprinzip gelöst werden - Das macht Max.\\
	Anders geht es über eine widerliche Sache - Das macht Aaron.
\end{Answer}