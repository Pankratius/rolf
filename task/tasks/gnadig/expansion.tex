\begin{Exercise}[origin = 200 Puzzling Physics Problems, title = Temperaturabhängige Wärmekapazität, difficulty = 4, label = expansion]
Wir betrachten einen beliebigen thermodynamischen Prozess, der mit $n$ Mol Heliumgas arbeitet. Desen Wärmekapazität sei gegeben durch $C = 3RT/(4T_0)$, wobei $T_0$ die Anfangstemperatur des Gases ist\footnote{Dabei ist die Wärmekapazität definiert als Änderungsrate der Wärme $\Delta Q$ bei Temperaturänderung $\Delta T$, $C = dQ/dT$}. Wie groß ist die verrichtete Arbeit bis das Gas sein minimales Volumen erreicht?
\end{Exercise}