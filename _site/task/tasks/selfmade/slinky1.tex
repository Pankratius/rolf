\begin{Exercise}[label = slinky1, origin = P.Gnädig, difficulty = 4, title = Fallendes Slinky]
	Ein \textit{Slinky} ist eine unausgedehnte, spiralförmige Feder, die in guter Näherung dem Hookschen Gesetz gehorcht.\\
	Betrachte nun ein solches Slinky der Masse $m$, welches ruhig auf einem Tisch liegt. Jemand zieht das obere Ende soweit nach oben, dass das untere Ende gerade noch liegen bleibt. Dabei ist die Länge des Slinkys $\ell$. Wie viel Arbeit muss zum Heben aufgebracht werden?
\end{Exercise}

	
