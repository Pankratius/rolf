\begin{Exercise}[difficulty = 3, origin = 3. Runde IPhO-Auswahlwettbewerb 2008, title = Bimetallstreifen, label = bms ]
	\hspace{4pt}
	Ein geklebter Bimetallstreifen besteht aus zwei Metallschichten der Dicke $d$, die  Wärmeausdehungskoeffizienten $\alpha_1$ bzw. $\alpha_2$ ($\alpha_2>\alpha_1$) haben. Im Anfgangszustand ist der Streifen gerade. 
	Wie groß ist der Krümmungsradius, wenn der Streifen um $\Delta T$ erwärmt wird? Was passiert im Grenzfall fast gleicher Wärmeausdehungskoeffizienten ($\alpha_2\rightarrow\alpha_1$)?
\end{Exercise}
