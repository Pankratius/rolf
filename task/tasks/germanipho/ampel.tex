\begin{Exercise}[label = ampel, title = Gute Ampel, difficulty = 3, origin = Alte IPhO-Aufgabe ]
	Ein Auto nähert sich auf einer Straße einer Ampel mit der Geschwindigkeit $v_0$.  Der Gleitreibungskoeffizient zwischen Straße und Auto beträgt $\mu$.\\ Die Ampel springt von Grün auf Gelb, wenn der Abstand Auto-Ampel gerade $s_0$ beträgt. Wie lang muss die Ampel gelb sein, damit der Autofahrer, unabhängig von $v_0$, entweder vor der Ampel zum Stehen kommt, oder aber mit konstanter Geschwindigkeit die Ampel kreuzt?
\end{Exercise}	
\begin{Answer}[ref = ampel]
	Diese Aufgabe kann man über mehrere Wege lösen.\\
	 Am Einfachsten geht es mit der Energieerhaltung. Wenn das Auto nach zurücklegen der Strecke $s_0$ auf jeden Fall zum Stehen gekommen sein, muss die auf dieser Strecke aufgebrachte Reibungsarbeit $E_r$ mindestens so groß sein, wie die kinetische Energie $E_{kin}$ vor dem Bremsen. Damit gillt
	 \begin{equation}\label{energycond}
	 E_{r} \geq E_{kin} \Rightarrow mg\mu s_0 \geq \frac{mv_0^2}{2}.
	 \end{equation}
	 Damit man vor der Ampel bei gegebener Strecke stehen bleiben kann, muss die Geschwindigkeit so gewählt werden, dass \eqref{energycond} erfüllt ist. Durch das Wurzelziehen ändert sich die Richtung der Ungleichung nicht, sodass man höchsten mit einer Geschwindigkeit von 
	 \begin{equation}\label{vfriccond}
	 	v_0\leq \sqrt{2s_0g\mu}
	 \end{equation}
	 vor der Ampel stehen bleiben kann.\\
	 Wenn man die Ampel bei gegebener Gelbphase ohne zu Bremsen kreuzen will, muss die Geschwindigkeit $v_0$ dafür groß genug sein, also
	 \begin{equation}\label{velcond}
	 	v_0 \geq  \frac{s_0}{t_g}
	 \end{equation}
	Da ein Bremsen aber immer möglich seien soll, können wir beide Bedingungen zusammen in einer Ungleichung schreiben
	\begin{equation}\label{stvcond}
		\sqrt{2s_0g\mu}\geq v_0 \geq \frac{s_0}{t}.
	\end{equation}
	Nach Aufgabenstellung soll die Zeit der Gelbphase $t_g$ so gewählt werden, dass die Bedingung \eqref{stvcond} unabhängig von der Geschwindigkeit ist. Das ist der Fall, wenn
	\begin{equation}\label{energytcond}
	\boxed{
	\sqrt{2s_0g\mu} \leq \frac{s_0}{t_g} \Rightarrow t_g \geq  \sqrt{\frac{s_0}{2g\mu}}}
	\end{equation}
	gilt.\\
	%Man kann die Aufgabe auch ohne Energieerhaltung nutzen. Dann muss man drei Bedingungen betrachten. Die erste Bedingung sagt, dass das Auto zum stehen gekommen ist, bevor die Ampel umschaltet, seine Geschwindigkeit also kleiner gleich null ist
	%\begin{equation}\label{traff:cond1}
%		v\left(t_g\right)\leq 0 \Rightarrow v_0 - g\mu t_g \leq 0 \Rightarrow v_0 \leq g \mu t_g.
%	\end{equation}
%	Gleichzeitig muss die Strecke, die das Auto in der Zeit $t_g$ bremsend zurücklegt, kleiner sein, als der Abstand vom der Ampel. Weil es sich um eine gleichmäßig beschleunigte Bewegung mit der Anfangsgeschwindigkeit $v_0$ handelt, muss also gelten
%	\begin{equation}\label{traff:cond2}
%	s\left(t_g\right) \leq s_0 \Rightarrow	-\frac{g\mu}{2}t_g^2+v_0t_g \leq s_0 \Rightarrow v_0 \leq \frac{s_0+\frac{g\mu}{2}t_g^2}{t_g}.
%	\end{equation}
%	Für den Fall, dass das Auto nicht bremst, sondern mit konstanter Geschwindigkeit weiterfährt, muss wieder gelten
%	\begin{equation}\label{traff:cond3}
%		v_0 \geq \frac{s_0}{t_g}.
%	\end{equation}
%	Wir können jetzt zuerst 
\end{Answer}