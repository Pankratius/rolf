\documentclass[a4paper]{article}
%\usepackage{exercise}
%um nur aufgaben zu zeigen
\usepackage[noanswer]{exercise} 
\usepackage{../images/orpheuspreamble}
\usepackage{rotating}
\usetikzlibrary{decorations.pathmorphing}
\usetikzlibrary{decorations.markings}
\usetikzlibrary{arrows}
\usetikzlibrary{shapes.geometric}
\newcommand{\midarrow}{\tikz \draw[-triangle 90] (0,0) -- +(.02,0);}
\usepackage{xcolor}
%\usepackage{draftwatermark}
%\SetWatermarkText{\textsc{Draft 2}}
%\SetWatermarkScale{3}
%\SetWatermarkColor{red!30}

\usepackage[printwatermark]{xwatermark}
%\newsavebox\mybox
%\savebox\mybox{\tikz[color=red,opacity=0.3]\node{\textsc{Entwurf}};}
%\newwatermark*[
%allpages,
%angle=45,
%scale=10,
%xpos=-4cm,
%ypos=4cm
%]{\usebox\mybox}
\pagestyle{fancy}
\fancyhead[L]{\includegraphics[width=2cm]{../images/logo_scaled.pdf}}
\fancyhead[R]{\textsc{Aufgabenseminar Mechanik}}

\begin{document}
	\vspace*{-1cm}
	\parbox{4cm}{\vspace{-0.2cm}\includegraphics[width=5cm]{../images/logo_scaled.pdf}}
	\parbox{12.5cm}{\setstretch{2.0} \centering{ \huge \textsf{Aufgabenseminar\\klassische Mechanik 
			}}\\
			\url{pankratius.github.io/rolf} \\ \vspace*{-.5cm} }
		\vspace{0.5cm}

\thispagestyle{empty}

\noindent
\begin{Exercise}[label = Looping, origin = Maximilian Marienhagen, difficulty = 1, title = Looping]
In welcher Höhe $h$ muss eine Kugel losrollen, um einen Looping mit Radius $R$ durchqueren zu können?  
\end{Exercise}
\begin{Exercise}[label = Bonbon, origin = {4. Runde zur 48. IPhO 2017}, title = Bonbon, difficulty = 2]
Schätze i Abhängigkeit von relevanten Parametern ab, wie lange man braucht, um ein kugelförmiges Bonbon der Masse $m$ zu lutschen.\\
Wie verhalten sich die Zeiten zueinander, wenn man zwei Bonbons betrachtet, von denen eines doppelt so schwer ist, wie das andere? Wie wenn eines einen doppelt so großen Radius hat wie das andere?
\end{Exercise}

\begin{Exercise}[label = Brunnen, origin = {Maximilian Marienhagen}, title = Brunnen, difficulty = 2]
Ein Stein wird in einen Brunnen fallengelassen. Nach einer Zeit $t$ hört man oben am Brunnenschacht das Auftreffen des Steins. Die Schallgeschwindigkeit sei $c$.\\
Wie tief ist der Brunnen?
\end{Exercise}

\begin{Exercise}[label = Motorrad in der Kugel, origin = Maximilian Marienhagen, difficulty = 2, title = Motorrad in der Kugel]Betrachte eine Kugel mit Radius $R$ und einen (natürlich punktförmigen) Motorradfahrer, der mit einer Geschwindigkeit $v$ in einem horizontalen Kreis auf der Innenseite der Kugel fährt.\\ In welcher Höhe $h$ über dem Boden ist er unterwegs? (Reibung darf vernachlässigt werden.)
\end{Exercise}
\begin{Exercise}[label = Kugelschuss, origin = IPHO 1967, difficulty = 2, title = Kugelschuss]
Auf einer Säule der Höhe $h$ liegt eine Kugel der Masse $M$. Völlig grundlos wird sie von einem Projektil der Masse $m$ durchschossen, welches vorher mit der Geschwindigkeit $v$ unterwegs war.
\begin{enumerate}
\item In welcher Entfernung $p$ trifft das Projektil am Boden auf, wenn die Kugel in der Entfernung $k$ aufkommt?
\item Welcher Teil der kinetischen Energie wird bei diesem Vorgang in Wärme umgewandelt?
\end{enumerate} 
\end{Exercise}
\begin{Exercise}[label = Brett auf Rollen, origin = Vorbereitungsaufgabe für die
4. Runde zur 47. IPhO 2016, difficulty = 3, title = Brett auf Rollen]
Gegeben seien zwei parallele Zylinder mit identischem Radius $R$ und Abstand $L$ voneinander, die sich in entgegengesetzte Richtungen drehen. Auf ihnen liegt ein Brett der Masse $m$. Der Reibungskoeffizient zwischen Brett und Zylinder ist $\mu$.\\
Gib eine Gleichung für die Bewegung des Bretts an.  
\end{Exercise}
\begin{Exercise}[label = Zylinder und Quader, origin = IPHO 1968, difficulty = 3, title = Zylinder und Quader]
Auf einer schiefen Ebene mit Neigungswinkel $\alpha$ liegt ein homogener Zylinder der Masse $m$, der durch eine inelastische Schnur mit einem Quader (Masse $M$ und Reibungskoeffizient $\mu$) verbunden ist.\\ Wie groß ist die Beschleunigung dieses Systems?
\end{Exercise}
\input{../tasks/germanipho/flummis}
\begin{Exercise}[label = Trägheitsmoment, origin = {4. Runde zur 44. IPhO 2013}, title = Trägheitsmoment, difficulty = 2]
Eine homogene Platte der Dicke $d$ und Masse $m$ besitze die Form eines gleichseitigen Dreiecks der
Seitenlänge $a$.\\
Berechnen Sie das Trägheitmoment der Platte bei Drehung um die zur Platte senkrechte Schwerpunktsachse.\\
Hinweis: Es ist möglich (aber nicht zwingend erforderlich) diese Aufgabe ohne Integration zu lösen.
\end{Exercise}

\begin{Exercise}[label = adi, title = Pendel, difficulty = 5, origin = Aaron Wild]
	Wir betrachten ein harmonisches Pendel, desen Länge sich langsam ändert. Wie ändert sich die Amplitude der Schwingung?
\end{Exercise}
\begin{Exercise}[label = grassh, title = fauler Grasshüpfer, origin = P.Gnädig, difficulty = 5]
	Ein fauler Grasshüpfer möchte über einen Baumstumpf mit Radius $r = 20~\mathrm{cm}$ springen.\\
	Wie groß ist die dafür mindestens benötigte Geschwindigkeit, wenn der Luftwiderstand vernachlässigt werden kann?
\end{Exercise}
\begin{Exercise}[origin = Jaan Kalda, title = Fuchsjagd, difficulty = 4, label = fox]
	Ein Hund jagt einen Fuchs, welcher sich mit einer konstanten Geschwindigkeit $v$ entlang einer Geraden bewegt. Der Hund bewegt sich ebenfalls mit $v$, jedoch ist sein Geschwindigkeitsvektor $\mathbf{v}$ immer nach dem Fuchs ausgerichtet. Am Anfang befindet sich der Hund senkrecht zu dem Fuchs, und die beiden haben einen Abstand von $\ell$. Was ist der minimale Abstand zwischen den beiden während der Verfolgung?
\end{Exercise}
\begin{Exercise}[origin = Jaan Kalda, difficulty = 4, title = Keil, label = wedge1]
	Wir betrachten eine kleine Kugel der Masse $m$, welche auf einem Keil der Masse $M$ mit Neigungswinkel $\beta$ liegt. Dabei ist die Kugel mit einem masselosen Faden und einer (ebenfalls masselosen Rolle) an der Wand befestigt. Finde die Geschwindigkeit des Keils.
\end{Exercise}

\begin{Exercise}[title = Flugzeuge, origin = Jaan Kalda]
	Zwei Flugzeuge fliegen auf gleicher Höhe mit den Geschwindigkeiten $v_1 = 600~\mathrm{\frac{km}{h}}$ und $v_2 = 800~\mathrm{\frac{km}{h}}$. Dabei befinden sie sich am Anfang auf den Eckpunkten eines gleichschenklig-rechtwinkligen Dreiecks mit der Seitenlänge $a = 20~\mathrm{km}$. Berechne den geringsten Abstand zwischen den beiden Flugzeugen unter der Annahme konstanter Geschwindigkeit.
\end{Exercise}

\begin{Exercise}[title = Wassertropfen, origin = Lukas Rettenmeier]
	Eine Wolke besteht aus einer Ansammlung von sehr kleinen Wassertröpfechen, die homogen im Raum verteilt sind. Nun fällt ein großer Wassertropfen durch die Wolke, wobei der Tropfen das Wasser jedes Tröpfchens, mit dem er zusammenstoßt, in sich aufnimmt. Nach einer langen Zeit wird sich der Tropfen mit einer konstanten Beschleunigung bewegen. Wie groß ist diese?
\end{Exercise}

\end{document}