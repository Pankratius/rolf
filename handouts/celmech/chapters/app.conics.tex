\section{Äquivalente Formen der Bahngleichung}
Wir betrachten die Gleichung
\[
r = \frac{p}{1+\varepsilon \cos \theta},
\]
wobei $\varepsilon,p\in \mathbb{R}^{\geq 0}$ gelte. Wir formen um und quadrieren
\begin{align*}
& r\left(1+\varepsilon \cos \theta\right) = p \\
\implies& r = p - r \varepsilon \cos \theta \\
\implies& r^2 = p^2 - 2pr\varepsilon\cos \theta + r^2 \varepsilon^2 \cos^2\theta
\end{align*}
Setzen wir jetzt wieder $x=r\cos \theta$ und $y=r\sin \theta$, dann erhalten wir
\begin{equation}\label{app:conic1}
 x^2(1-e^2)+2epx+y^2=p^2
\end{equation}
Wir betrachten unterschiedliche Fälle
\paragraph*{Fall 1: $\varepsilon = 0$}
Wir erhalten
\[
x^2+y^2=p^2.
\]
Es handelt sich bei der Bahn also um eine Kreisbahn.
\paragraph*{Fall 2: $0<\varepsilon<1$}
Wir definieren $\overline{x}$ durch Translation von $x$
\[
\overline{x}:= x+ \frac{\varepsilon p}{1-\varepsilon^2},
\]
sodass \eqref{app:conic1}
die Form
\begin{equation}\label{app:conicel}
\frac{\overline{x}^2}{a^2}+\frac{y^2}{b^2}=1
\end{equation}
mit
\[
a^2 := \frac{p^2}{(1-\varepsilon^2)^2},~b^2:= \frac{p^2}{1-\varepsilon^2}
\]
hat. Dabei handelt es sich um eine Ellipse. Also ist die ursprünglich Bahn ebenfalls Ellipse, die lediglich auf der $x$-Achse verschoben war.
\paragraph*{Fall 3: $\varepsilon>1$}
Wir betrachten die gleiche Translation wie oben, setzen aber diesmal
\[
b^2:= \frac{p^2}{e^2-1},
\]
und erhalten bringen damit \eqref{app:conic1} auf die Form
\begin{equation}\label{app:conichyp}
  \frac{\overline{x}^2}{a^2}-\frac{y^2}{b^2}=1
\end{equation}
Also ist die Bahn hier durch eine auf der $x$-Achse verschobene Hyperbel gegeben:
\paragraph*{Fall 4: $\varepsilon=1$}
In diesem Fall können wir die Translation
\[
\overline{x}:= x-\frac{p}{2}
\]
betrachten.
Dann erhalten wir aus \eqref{app:conic1}
\begin{equation}\label{app:conicpar}
  y^2+2p\overline{x} = 0
\end{equation}
Gleichung \eqref{app:conic1} beschreibt also hier eine auf der $x$-Achse verschobene Parabel.
