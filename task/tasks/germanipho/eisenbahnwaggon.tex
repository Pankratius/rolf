	\begin{Exercise}[label=ipho201741, title = Eisenbahnwaggon,difficulty={1},origin = 4. Runde IPhO 2017]
 Ein Eisenbahnwaggon der Masse $m_1 = 37~\mathrm{t}$ rollt reibungsfrei auf einem Bahngleis und stößt mit einem anderen Waggon zusammen. Beide verkuppeln sich und rollen zusammen weiter. Dabei geben sie 35\% der anfangs vorhandenen kinetischen Energie als Wärme ab. Wie groß ist die Masse $m_2$ des zweiten Waggons?
	\end{Exercise}
\begin{Answer}[ref = ipho201741]
	Da keine äußeren Kräfte wirken, bleibt der Impuls erhalten.\\ Gleichzeitig bleibt die Gesamtenergie erhalten,
	\begin{equation}\label{energycons}
		E_{kin,v} = E_{kin,n}+E_{w}.
	\end{equation}
	Hierbei ist $E_{kin,v}$ die kinetische Energie vor dem Stoß, $E_{kin,n}$ die kinetische Energie nach dem Stoß, und $E_{w}$ die abgegebene Wärmeenergie. Es ist gegeben, dass $E_{w} = \eta \cdot E_{kin,v}$ gilt, wobei $\eta = 0.35$ gilt. Um die Impulserhaltung zu benutzten kann man entweder die kinetische Energie durch die Impulse oder aber durch die Geschwindigkeiten ausdrücken.\\
	Drückt man die kinetische Energie durch die Impulse aus, so ist im Allgemeinen $E_{kin} = \frac{p^2}{2m}$. Da der Impulserhalten bleibt, gilt $p:=p_v = p_n = \mathrm{konst.}$. Damit lässt sich die Energieerhaltung schreiben als
	\begin{equation}\label{penenergycons}
		\frac{p^2}{2m_1}= \frac{p^2}{2\left(m_1+m_2\right)} + \eta\cdot \frac{p^2}{2m_1},
	\end{equation}
	da nach dem Stoß nur noch ein \glqq Waggon\grqq{} der Masse $m_1+m_2$ da ist.
	Division durch $p^2$ auf beiden Seiten führt auf
	\begin{equation}\label{masscond}
		\frac{1}{2m_1}= \frac{1}{2\left(m_1+m_2\right)}+\frac{\eta}{2m_1}.
	\end{equation}
	Diese Gleichung erhält nur noch die bekannten Größen $m_1$ und $\eta$, sowie die unbekannte Größe $m_2$. Subtrahieren von $\frac{\eta}{2m_1}$, multiplizieren mit 2 und anschließendes Bilden des Reziproken führt auf
	\begin{equation*}\label{solp}
	\boxed{
		\frac{1}{m_1}\left(1-\eta\right) = \frac{1}{m_1+m_2} \Rightarrow m_2 = m_1\left(\frac{1}{1-\eta}-1\right)\approx 20~\mathrm{t}.}
	\end{equation*}
	Das gleiche kann man auch mit den Geschwindigkeiten machen. Hier hat die Energieerhaltung die Form
	\begin{equation}\label{venergycons}
		m_1 \frac{v_v^2}{2} = \left(m_1+m_2\right)\frac{v_n^2}{2} + m_1\eta  \cdot \frac{v_v^2}{2}.
	\end{equation}
	Expliziet ausgeschrieben besagt die Impulserhaltung, dass 
	\begin{equation}\label{secondvel}
		m_1 v_v = \left(m_1+m_2\right)v_n \Rightarrow v_n = \frac{m_1}{m_1+m_2}v_v
	\end{equation}
	gilt. Einsetzen von \eqref{secondvel} in \eqref{venergycons} führt auf
	\begin{equation*}
	m_1\frac{v_v^2}{2} = \left(m_1+m_2\right)\frac{v_v^2 m_1^2}{2\left(m_1+m_2\right)^2} + m_1\eta \cdot \frac{v_v^2}{2}.
	\end{equation*}
	Dividieren durch $m_1^2v_v^2$ führt auf 
	\begin{equation}\label{vmasscond}
		\frac{1}{2m_1} = \frac{1}{2\left(m_1+m_2\right)} + \frac{\eta}{2m_1}.
	\end{equation}
	Die Bedingung, die die Massen erfüllen müssen, wenn wir Geschwindigkeiten betrachten (Gleichung \eqref{vmasscond}) entspricht also genau der, die erfüllt sein muss, wenn wir Impulse betrachten (Gleichung \eqref{masscond}). Das sollte auch so sein.
\end{Answer}
