\begin{Exercise}[label = snowploughing, title = Schneepflug, origin = {STEP 1987, Paper 3}, difficulty = 4]
	Zwei identische Schneepflüge räumen die gleiche Straße. Der erste startet eine Zeit $t_1$ nach dem es mit scheinen anfing, der zweite vom gleichen Punkt nach einer Zeit $t_2-t_1$.\\
	Der Schnee fällt so, dass sich die Höhe der Schneedecke mit einer kontanten Rate $k$ vergrößert.\\
	Die Geschwindigkeit einer Schneeraupe ist $v\left(t\right) = a k/z\left(t\right)$, wobei $z\left(t\right)$ die momentante Schneehöhe angibt, und $a$ eine Konstante ist.\\
	Jeder Schneepflug räumt den gesamten Schnee. Zeige, dass die Zeit $t$, zu der der zweite Schneepflug eine Distanz $x_2\left(t\right)$ zurückgelegt hat, die Gleichung
	\begin{equation}\label{snowp:invertedeom}
		a \frac{dt}{dx_2} = t - t_1e^{\left(\nicefrac{x_2}{a}\right)}
	\end{equation}
	erfüllt.\\
	Bestimme mit \eqref{snowp:invertedeom} die Zeit, bis die beiden Schneepflüge kollidieren.
\end{Exercise}