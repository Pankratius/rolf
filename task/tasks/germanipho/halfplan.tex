\begin{Exercise}[label = halfplan, title = Ein halber Planet, origin = {4.Runde IPhO 2015, Kurzaufgabe}, difficulty = 3]
	Auf einem halbkugelförmigen Planeten beträgt die Schwerebeschleunigung direkt in der Mitte der flachen Deckfläche $g_0$.  Die Dichte ist im gesamten Planeten gleich, und beträgt $\rho$. Wie groß ist der Planetenradius?
\end{Exercise}
\begin{Answer}[ref = halfplan]
	Wir betrachten den Planten, also die Halbkugel, als durch viele sehr kleine Schichten der Dicke $dr$ aufgebaut.\\
	Das 
\end{Answer}