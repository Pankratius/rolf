\documentclass[a4paper]{article}
%\usepackage{exercise}
%um nur aufgaben zu zeigen
\usepackage[noanswer]{exercise} 
\usepackage{../images/preamble}
\usepackage{rotating}
\usetikzlibrary{decorations.pathmorphing}
\usetikzlibrary{decorations.markings}
\usetikzlibrary{arrows}
\newcommand{\midarrow}{\tikz \draw[-triangle 90] (0,0) -- +(.02,0);}

\begin{document}
	\vspace*{-2cm}
	\parbox{4cm}{\includegraphics[width=2.5cm]{../images/ROLF4.png}}
	\parbox{10.6cm}{\setstretch{2.0} \centering{ \huge \textsf{Aufgabenserie 3 - Sommerspaß}}\\ Abgabe: 17. August \\ \vspace*{-.5cm} }
	
	

\pagestyle{empty}
\begin{framed}
	\noindent
	\scriptsize
	Diesmal gibt es vier Aufgaben, dafür auch zwei statt nur einem Monat. \textsc{ROLF wünscht viel Spaß in den Sommerferien.}\\
	Die Aufgaben sollten bis zum \textbf{17. August} bearbeitet werden. Die Lösungen schickt ihr entweder an \href{mailto:physikrolf@gmail.com}{physikrolf@gmail.com}, oder per Post an: Aaron Wild, Zum wilden Graben 26, 99425 Weimar.
	Jede Aufgabe hat eine bestimmte Anzahl an erreichbaren Punkten. Wie viele das sind, müsst ihr raten. Versucht, die Lösungen so genau wie möglich aufzuschreiben. Für besonders schnelle/gute/witzige Lösungen kann es Bonuspunkte geben.\\ Die aktuellen Aufgaben sowie alle alten Aufgabenserien mit Lösungen findet ihr auch auf \url{pankratius.github.io/rolf}.\\ \textit{Zusätzlich veröffentlichen wir hier Mitte Juli Tipps zu allen vier Aufgaben!}
\end{framed}

\noindent

\begin{Exercise}[label = grassh, title = fauler Grasshüpfer, origin = P.Gnädig, difficulty = 5]
	Ein fauler Grasshüpfer möchte über einen Baumstumpf mit Radius $r = 20~\mathrm{cm}$ springen.\\
	Wie groß ist die dafür mindestens benötigte Geschwindigkeit, wenn der Luftwiderstand vernachlässigt werden kann?
\end{Exercise}

\begin{Exercise}[label = infnet, title = Widerstandsnetzwerk, origin = P.Gnädig, difficulty = 4]
	Alle Ecken in dem unendlich großen Widerstandsnetz (siehe Abbildung) haben den Widerstand $R$. Wie groß ist der Widerstand zwischen $A$ und $B$?
\end{Exercise}

\begin{figure}[h]
	\centering
	\begin{tikzpicture}
		\draw
		(-2,1) to[/tikz/circuitikz/bipoles/length=20pt, R] (0,1)
		(0,1) to[/tikz/circuitikz/bipoles/length=20pt, R](1,1)
		(1,1) to[/tikz/circuitikz/bipoles/length=20pt, R](3,1)
		(-1.75,1)to[/tikz/circuitikz/bipoles/length=20pt, R](-1.75,0)
		(-1.75,0)to[/tikz/circuitikz/bipoles/length=20pt, R](-1.75,-1)
				(-2,0) to[/tikz/circuitikz/bipoles/length=20pt, R] (0,0)
				(0,0) to[/tikz/circuitikz/bipoles/length=20pt, R](1,0)
				(1,0) to[/tikz/circuitikz/bipoles/length=20pt, R](3,0)
						(-2,-1) to[/tikz/circuitikz/bipoles/length=20pt, R] (0,-1)
						(0,-1) to[/tikz/circuitikz/bipoles/length=20pt, R](1,-1)
						(1,-1) to[/tikz/circuitikz/bipoles/length=20pt, R](3,-1)
				(-.25,1)to[/tikz/circuitikz/bipoles/length=20pt, R](-.25,0)
				(-.25,0)to[/tikz/circuitikz/bipoles/length=20pt, R](-.25,-1)
					(1.25,1)to[/tikz/circuitikz/bipoles/length=20pt, R](1.25,0)
					(1.25,0)to[/tikz/circuitikz/bipoles/length=20pt, R](1.25,-1)
					(2.75,1)to[/tikz/circuitikz/bipoles/length=20pt, R](2.75,0)
					(2.75,0)to[/tikz/circuitikz/bipoles/length=20pt, R](2.75,-1)
			
					;
		\draw (2.75,1) -- (2.75,1.2);
		\draw (-1.75,1) -- (-1.75,1.2);
		\draw (2.75,-1) -- (2.75,-1.2);
		\draw (-1.75,-1)--(-1.75,-1.2);
		\filldraw[black] (-.25,0) circle (1.5pt);
		\filldraw[black] (1.25,0) circle (1.5pt);
		\node at (0,-.2) {$A$};
		\node at (1.5,-.2) {$B$};
		\node at (0.5,0.25){$R$};
		\node at (-2.2,1) {$...$};
			\node at (-2.2,0) {$...$};
				\node at (-2.2,-1) {$...$};
		\node at (3.2,1) {$...$};
			\node at (3.2,0) {$...$};
			\node at (3.2,-1) {$...$};
		\node at (-1.75,-1.5) {$...$};
		\node at (-0.25,-1.5) {$...$};
		\node at (1.25,-1.5) {$...$};
		\node at (2.75,-1.5) {$...$};
				\node at (-1.75,1.5) {$...$};
				\node at (-0.25,1.5) {$...$};
				\node at (1.25,1.5) {$...$};
				\node at (2.75,1.5) {$...$};
		
		

	\end{tikzpicture}
\end{figure}
\begin{minipage}[b]{0.7\textwidth}
\begin{Exercise}[label = col1, title = Stoßaufnahme, origin = Aaron Wild, difficulty = 5]
	Die nebenstehende Abbildung zeigt die Positionen von zwei Körpern zum Zeitpunkt $t_0$ und $t_1$. Die Körper sind sehr klein, und bewegen sich reibungsfrei auf einem Tisch.\\
	Der rote Körper hat eine Masse, die dreimal so groß ist wie die des blauen.\\
	Bestimme in der Abbildung (in größerer Form auch auf der nächsten Seite) die Bewegungsrichtungen der beiden Körper, nachdem sie zusammengestoßen sind.
\end{Exercise}
\end{minipage}
\begin{minipage}[b]{0.3\textwidth}
	\flushright
	\begin{tikzpicture}
	\filldraw[color =black, fill = blue]  (-2,2.31) circle (2pt);
	\filldraw[color = black, fill = blue] (0,1.155) circle (2pt);
	\filldraw[color = black, fill = red] (0,0) circle (2pt);
	\filldraw[color = black, fill = red] (1,0) circle (2pt);
	\node at (-2,2.7) {$t_0$};
	\node at (0,1.5) {$t_1$};
	\node at (0,-0.5) {$t_0$};
	\node at (1,-0.5){$t_1$};
\end{tikzpicture}
\end{minipage}
\vspace{2cm}
\begin{minipage}[b]{0.6\textwidth}
\begin{Exercise}[title = Variabler Brechungsindex, label = varn1, origin = IPhO 1974, difficulty = 4]
	Gegeben ist eine planparalle Platte. Ihre Brechzahl ändert sich nach der Gleichung
	\begin{equation}\label{varn1:nx}
		n\left(x\right) = \frac{n_a}{1-\frac{x}{q}},
	\end{equation}
	wobei $n_a = 1.2$ die Brechzahl im Punkt $A$, und $q = 0.13~\mathrm{cm}$ eine Konstante ist. \\
	Im Punkt $A$ ($x_a = 0$) fällt senkrecht zur Platte ein Lichstrahl ein. Dieser verlässt die Platte im Punkt $B$ unter einem Winkel von $\alpha = 30^{\circ}$ zur ursprünglichen Richtung. Bestimme den Brechungsindex des Materials im Punkt $B$ und die Dicke der Platte $d$.
\end{Exercise}
\end{minipage}
\begin{minipage}[b]{0.4\textwidth}
\centering
\begin{tikzpicture}
\clip (-1.5,-2.75) rectangle (4,4);
\draw[thick] (-0.5,1) -- (2.75,1);
\draw[thick] (-0.5,-1) -- (2.75,-1);
\draw[decorate,decoration={snake, amplitude = 0.7mm, segment length = 5.5mm}] (-0.5,1) -- (-0.5,-1);
\draw[decorate,decoration={snake, amplitude = 0.7mm, segment length = 5.5mm}] (2.5,1) -- (2.5,-1);
\draw[very thick,->] (0,-1) -- (0,2);
\draw[very thick,->] (0,-1) -- (3,-1);
\node at (-0.25,1.9) {$y$};
\node at (3.2,-1) {$x$};
\draw[<->] (-1,1) -- (-1,-1) node[midway, fill = white] {$d$};
	\begin{scope}[ every node/.style={sloped,allow upside down}
	] 
			\draw(0,-1.75) -- node {\midarrow} (0,-1);
	\end{scope}
\node at (-0.2,-1.2) {$A$};
\node at (1.5,0.7) {$B$};
\tkzDefPoint(1.5,1){B}
\tkzDefPoint(0,-1){A}
\tkzDefPoint(1.5,1.86){b1}
\tkzDefPoint(2,1.866025){b2}
\tkzMarkAngle[scale = 0.75](b2,B,b1)
\tkzLabelAngle(b2,B,b1){$\alpha$}
\tkzDrawSegment(B,b1)
\tkzDrawSegment(B,b2)
\tkzDefPoint(0.9,0.4){C}
\tkzCircumCenter(A,C,B)\tkzGetPoint{O}
\tkzDrawArc(O,B)(A)
\end{tikzpicture}
\end{minipage}
\begin{Answer}[ref = varn1]
	Wir rechnen zuerst den Brechungsindex des Materials im Punkt $B$ aus. Dazu hilft uns das Brechungsgesetz. Allgemein gilt für einen Strahl, der aus einem Material mit Brechungsindex $n_1$ unter einem Winkel $\alpha_1$ zum Lot auf einen Übergang zu einem Material mit Brechungsindex $n_2$ triftt
	\begin{equation*}
		n_1 \sin \alpha_1 = n_2 \sin \alpha_2,
	\end{equation*}
	wobei $\alpha_2$ der entsprechende Ausfallswinkel ist.\\
	Wir können uns das Material mit dem sich änderten Brechungsindex so vorstellen, als sei es aus sehr vielen kleinen Schichten mit Brechungsindex $n_i$ aufgebaut. Da jeweils das Brechungsgesetz gilt, muss also für zwei direkt hintereinader liegende Schichten gelten:
	\begin{equation*}
		n_{i-1} \sin \alpha_{i-1} = n_{i} \sin \alpha_i.
	\end{equation*}
	Das ist aber nichts anderes als die Aussage, dass $n \sin \alpha := c$ entlang des gesamten Materials konstant bleiben muss.\\
	Damit können wir jetzt den Brechungsindex im Punkt $B$ ausrechnen. Weil das Licht senkrecht eintrifft, ist
	\begin{equation}\label{varn1:cdef}
	c = n_a \sin 90^\circ = n_a.
	\end{equation}
	Gleichzeitig gilt für die Brechung am Punkt $B$ 
	\begin{equation}\label{varn1:bdef}
	\sin \alpha = n_B \sin \left(90^\circ-\beta\right) = n_B \cos \beta ,
	\end{equation}
	wobei $\beta_b$ der Einfallswinkel im Punkt $B$ ist. Mit dem trigonmetrische Pythagoras ($\sin^2 x + \cos^2 x = 1$) und $c = n_B \sin \beta_B$ kann man \eqref{varn1:bdef} umstellen zu
	\begin{multline}\label{varn1:nb}
		\sin \alpha = n_b \sqrt{1-\sin^2 \beta_B} =  \sqrt{n_b^2-n_b^2 \sin^2 \beta_B} = \sqrt{n_b^2-c^2} \overset{\eqref{varn1:cdef}}{=} \sqrt{n_b^2-n_a^2}\\ \boxed{\Rightarrow n_b = \sqrt{\sin^2 \alpha + n_a^2} = 1.3.}
	\end{multline}
	Wir können jetzt die Form des Lichstrahls ausrechnen, und damit $d$ berechnen. Das geht über mehrere Methoden.\\
	Bei der ersten denkt man ein bisschen nach. Dazu betrachtet man den Einfallswinkel $\beta\left(x\right)$ bei dem Übergang zwischen zwei gedachten Schichten an der Stelle $x$. Für diesen gilt mit \eqref{varn1:cdef} und \eqref{varn1:nx}
	\begin{equation}\label{varn1:circdef}
		\sin \beta\left(x\right) = \frac{c}{n\left(x\right)} \overset{\eqref{varn1:cdef}}{=} \frac{n_a}{n\left(x\right)} \overset{\eqref{varn1:nx}}{=} \frac{q-x}{q}.
	\end{equation}
	Wir betrachten nun den Kreis $k$ um den Punkt $O$, welcher die Koordinaten $\left(q,0\right)$ hat. Die Tangente an diesen Kreis an der Stelle $x$ schließt mit dem Kreis genau den Winkel $\sin \beta\left(x\right)$ ein, wie man im Dreieck $\Delta OCC'$ sehen kann. Weil der Strahl auch wenn er in das Material kommt tangential zu diesem Kreis verläuft, muss er innerhalb des Materials entlag dieser Kreisbahn verlaufen.\\
	Gleichzeitig können wir die $x$-Koordinate von $B$ einfach durch Umstellen von \eqref{varn1:nx} finden
	\begin{equation}\label{varn1:xb}
		n_b = \frac{n_a}{1-\frac{x}{q}}\Rightarrow x_B =q \left(1-\frac{n_a}{n_b}\right) = 1~\mathrm{cm}.
	\end{equation}
	Damit können wir jetzt im rechtwinkligen Dreieck $\Delta BB'O$ die Dicke $d$ ausrechnen
	\begin{equation}\label{varn1:dsol}
	\boxed{
		d = \sqrt{q^2-\left(q-x_b\right)^2} = 5~\mathrm{cm}.}
	\end{equation}
	Mann kann nachdenken auch durch Ableiten ersetzen. Mit $\tan\left(\arcsin x\right) = \frac{x}{\sqrt{1-x^2}}$ können wir aus \eqref{varn1:circdef} den Anstieg der Funktion $y\left(x\right)$ für den Lichtstrahl bestimmen
	\begin{equation*}
		\frac{\mathrm{d}y}{\mathrm{d}x} = \tan \beta = \frac{q-x}{\sqrt{q^2-\left(q-x\right)^2}}.
	\end{equation*}
	Das ist eine separierbare Differentialgleichung, die wir integrieren können
	\begin{equation}\label{varn1:inte}
		\int \mathrm{d}y = \int \frac{q-x}{\sqrt{q^2-\left(q-x\right)^2}}~\mathrm{d}x \Rightarrow y\left(x\right)= \sqrt{q^2-\left(q-x\right)^2}+c
	\end{equation}
	Aus $y\left(0\right)= 0$ folgt $c=0$. \eqref{varn1:inte} ist aber nur die Gleichung für einen Halbkreis in positiver $y$-Richtung um $\left(q,0\right)$. Damit kann $d$ genauso wie in \eqref{varn1:dsol} ausgerechnet werden.
	
\end{Answer}


%





\newpage
	\hspace{4cm}
	\vspace{-1cm}
\begin{turn}{270}

\begin{tikzpicture}[scale = 3]
	\filldraw[color =black, fill = blue]  (-2,12.31) circle (0.05);
	\filldraw[color = black, fill = blue] (0,11.155) circle (0.05);
	\filldraw[color = black, fill = red] (-0.1,10) circle (0.05);
	\filldraw[color = black, fill = red] (0.9,10) circle (0.05);

\end{tikzpicture}
\end{turn}



\end{document}
