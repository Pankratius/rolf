\begin{minipage}[b]{0.8\textwidth}
\noindent
\begin{Exercise}[label = can, title = {Büchse}, origin = {nach 3. Runde IPhO, 2011}]
	Bestimme die Position des Schwerpunkts $h_s$ einer gefüllten zylinderförmigen Büchse in Abhängigkeit der Füllhöhe $h_f$ und der relevanten Parameter. Nimm dafür an, dass die Büchse eine gleichmäßige Massenverteilung hat. 
\end{Exercise}
\end{minipage}
\begin{minipage}[b]{0.2\textwidth}
\centering
\begin{tikzpicture}
\draw (-0.5,1)--(-0.5,0)--(0.5,0)--(0.5,1);
\fill[pattern = north east lines] (-0.5,0.6)--(-0.5,0)--(0.5,0)--(0.5,0.6);
\filldraw[black] (0,0.3) circle (2pt);
\draw[<->] (0.7,0) -- (0.7,0.3) node[midway, right]{$h_s$} ;
\draw[<->] (-0.7,0) -- (-0.7,0.6) node[midway, left]{$h_f$};

\end{tikzpicture}
\end{minipage}
