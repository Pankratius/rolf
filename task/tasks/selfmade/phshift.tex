\begin{Exercise}[label = viralthe, title = Periheldrehung, difficulty = 5, origin = Aaron Wild]
	Wir betrachten ein Gravitationsfeld, was vom normalen Keplerschen um eine sehr kleine Pertubation $\delta U\left(r\right)$ abweicht:
	\begin{equation}
		U^\dagger\left(r\right) = U\left(r\right) + \delta U\left(r\right) = U\left(r\right)+\frac{\beta}{r^2}
	\end{equation}
	Das führt dazu, dass die elliptische Bahn, auf der der Körper sich bewegt, nicht mehr geschlossen ist, sondern sich die Periapsis bei jeder Umdrehung um einen kleinen Winkel $\delta \varphi$ verschiebt. \\
	Bestimme $\delta \varphi$ in Abhängigkeit von den ursprünglichen Bahnparametern und $\beta$. 
	
\end{Exercise}
%\input{../tasks/selfmade/apoi.pdf}
