\begin{Exercise}[label = lso, difficulty = 3, origin = Aaron Wild, title = Schwarzschildmetrik]
	Die Bewegung eines Teilchens in der Schwarzschildmetrik\footnote[3]{Allgemein-relativistische Beschreibung der Gravitationswirkung eines ruhenden, sphärisch-symmetrischen Körpers.} kann durch ein effektives Potential der Form
	\begin{equation}\label{lso:ep}
		\tilde{U}\left(r\right) = \frac{mc^2}{2} \left(-\frac{r_s}{r} + \frac{a^2}{r^2} - \frac{r_sa^2}{r^3}\right).
	\end{equation}
	beschrieben werden. Dabei ist $r_s := \frac{2\Gamma}{c^2}$ der sog. \textit{Schwarzschildradius} und $ a := \frac{\ell}{mc}$ eine Referenzlänge.\\
	Die Gesamtenergie ist dann einfach
	\begin{equation}\label{lso:ene}
		E = \frac{m}{2}\dot{r}^2 + \tilde{U}\left(r\right).
	\end{equation}
	Finde das Intervall, in dem kreisförmige Orbits möglich sind, sowie den Abstand $r_{isco}$ vom Zentralkörper, an dem der letzte stabile kreisförmige Orbit möglich ist.
\end{Exercise}