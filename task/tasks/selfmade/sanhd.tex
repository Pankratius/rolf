\begin{Exercise}[label = sandh, origin = Aaron Wild, difficulty = 3, title = Sanduhr]
	Schätze mit geeigneten Methoden und Annahmen ab, die Zeit $t$ ab, die der Sand braucht, um nach dem Umdrehen durch eine handelsübliche Eieruhr zu fließen.
\end{Exercise}
\begin{Answer}
	Für diese Aufgabe ist es sinnvoll, wenn man sich zuerst überlegt, welche Parameter einen wichtigen Einfluss auf das Ergebnis haben.\\
	Weil der Sand durch die Erdanziehungskraft nach unten gezogen wird, ist sicherlich die Gravitationsbeschleunigung $g$ wichtig (eine Eieruhr geht auf dem Mond sicherlich anders als auf der Erde). Gleichzeitig macht es sicherlich einen Unterschied, ob die Eieruhr vor dem Umdrehen hoch oder weniger hoch gefüllt ist. Weil wir die anfängliche Füllhöhe $h$ auch gut messen können, ist es sinnvoll, auch diese Größe zu betrachten.\\
	Einen weiteren Einfluss wird es haben, ob das Loch, welches zwischen den Eieruhrhälften ist, groß oder klein ist. Das wird durch den Durchmesser $d$ bestimmt.\\
	Gleichzeitig kann es sein, dass die Dichte $\rho$ des Sandes auch einen Einfluss hat. Zudem erlaubt sie es, gemeinsam mit der Höhe $h$ die Gesamtmenge des Sandes abzuschätzen, wenn man die Zylinder der Sanduhr als Würfel annimmt.\\
	Was man jetzt machen kann, ist, sich zu fragen, wie man all die Größen, die uns wichtig erscheinen, so zusammenmultiplizieren kann, dass am Ende eine Zeit rauskommt.\\
	Wir nehmen also an, dass die Zeit ausschließlich eine Funktion der Gravtitationsbeschleunigung $g$, der Füllhöhe $H$, des Lochdurchmessers $d$ und der Sanddichte $\rho$ ist; $t = f\left(g,h,d,\rho\right)$. Die Frage ist also, wie $f\left(g,H,d,\rho\right)$ aussehen muss, damit es die Dimension der Zeit hat (also Sekunde, Minute, Stunde oder etwas vergleichbares).\\
	Dafür können wir uns anschauen, welche Dimensionen die gegeben Größen haben. Dabei schreiben wir die Dimension einer Größe $x$ als $[x]$. Für uns wichtig sollten hier die Dimensionen der Länge $[L]$, der Zeit $[T]$ und der Masse $[M]$ sein.\\
	Wir nehmen jetzt zuerst an, dass $f\left(g,h,d,\rho\right)$  einfach ein Produkt ist, weil wir sonst nur schwer eine Chance haben, $f$ zu finden.\\
	Dann fragen wir uns, mit welchen Exponenten die einzelnen Größen in diesem Produkt stehen müssen, damit am Ende die richtige Dimension da steht:
	\begin{equation}\label{sandh:dima}
			 t = g^\alpha\cdot h^\beta \cdot d^\gamma \cdot \rho\delta.
	\end{equation}
	Wir wollen jetzt die Exponenten $\alpha$, $\beta$, $\gamma$ und $\delta$ so rauszufinden, dass $[t] = [g^\alpha\cdot h^\beta \cdot d^\gamma \cdot \rho\delta] = [T]$ gilt. \\
	Dazu können wir uns zuerst 
\end{Answer}