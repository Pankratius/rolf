\begin{minipage}[b]{0.65\textwidth}
\begin{Exercise}[title = Polygon, origin = P. Gnädig, difficulty = 2, label = polygrav]
	Wir betrachten ein regelmäßiges $n$-Eck, bei dem an jeder Ecke eine Masse $m$ sitzt. Wie bewegt sich das System, wenn nur die Gravitationskraft zwischen den Körpern wirkt? Wie viel Zeit (in Abhänigkeit von $n$) vergeht, bis das System seinen Endzustand erreicht hat? 
\end{Exercise}
\end{minipage}
\begin{minipage}[b]{0.35\textwidth}
	\centering
	\begin{tikzpicture}[line cap=round,line join=round,>=triangle 45,x=1.0cm,y=1.0cm]
	\clip(-1.0911274272504616,-0.195261736026898) rectangle (1.5959243296533993,2.330040817448145);
	\draw(0.,0.) -- (1.,0.) -- (1.5,0.8660254037844387) -- (1.,1.7320508075688776) -- (0.,1.7320508075688779) -- (-0.5,0.8660254037844395) -- cycle;
	\draw (0.,0.)-- (1.,0.);
	\draw (1.,0.)-- (1.5,0.8660254037844387);
	\draw (1.5,0.8660254037844387)-- (1.,1.7320508075688776);
	\draw (1.,1.7320508075688776)-- (0.,1.7320508075688779);
	\draw (0.,1.7320508075688779)-- (-0.5,0.8660254037844395);
	\draw (-0.5,0.8660254037844395)-- (0.,0.);
	\draw(0.5,0.866025403784439) circle (1.cm);
	\draw [->] (0.5,0.866025403784439) -- (1.,1.7320508075688776);
	\begin{scriptsize}
	\draw [fill=black] (0.,0.) circle (2.0pt);
	\draw [fill=black] (1.,0.) circle (2.0pt);
	\draw [fill=black] (1.5,0.8660254037844387) circle (2.0pt);
	\draw [fill=black] (1.,1.7320508075688776) circle (2.0pt);
	\draw [fill=black] (0.,1.7320508075688779) circle (2.0pt);
	\draw [fill=black] (-0.5,0.8660254037844395) circle (2.0pt);
	\draw [fill=black] (0.5,0.866025403784439) circle (2.0pt);
	\end{scriptsize}
	\node at (0.5,1.4){$R_0$};
	\node at (0.5,0.5){$O$};
	\end{tikzpicture}
\end{minipage}
\begin{Answer}[ref = polygrav]
	Aus Symmetriegründen heben sich die nicht-radialen Teile der wirkenden Gravitationskräfte auf, sodass alle $n$ Massen sich zum Mittelpunkt des Polygons, $O$, bewegen. Dabei muss die polygonform erhalten bleiben. Weil die Abhängigkeit Körperabstand-Gravitationskraft aber nicht-linear ist, ist die Bewegung nicht gleichmäßig beschleunigt. Vielmehr sollte die Beschleunigung größer werden, je geringer der Körperabstand ist.\\
	Um nun die Zeit $T$ auszurechnen, bis die Körper im Punkt $O$ kollidieren, kann man zuerst die Kraft auf eine der Massen $m$ ausrechnen. Diese ist gegeben durch die Summe der Radialteile aller  Gravitationskräfte der anderen $n-1$ Körper, also, wenn der Radius des Polygons gerade $R$ ist,
	\begin{equation}\label{polygrav:f}
		F = Gm \sum_{i= 1}^{n-1}\frac{m \sin\left(\frac{\pi}{i}\right)}{\left(2R  \sin\left(\frac{\pi}{i}\right)\right)^2} = \frac{Gm}{R^2}\cdot \underbrace{ \frac{m}{4}\sum_{i=1}^{n-1} \frac{1}{\sin \frac{\pi}{i}}}_{:= M_n}.
 	\end{equation}
 	Die Kraft ist also so, als würde sich der Körper im Gravitationsfeld eines stationären Körpers mit der Masse $M_n = \frac{m}{4}\sum_{i=1}^{n-1} \frac{1}{\sin \frac{\pi}{i}}$ bewegen.\\
 	Diese Bewegung kann man jetzt als Bewegung entlang einer Ellipse ohne kleiner Halbachse, und mit großer Halbachse $\frac{R_0}{2}$ auffassen. Die Zeit, die dann bis zum Zusammensturz benötigt wird, entspricht genau der halben Periode $\nicefrac{T_e}{2}$. \\
 	Diese kann man über das dritte Keplersche Gesetz aus der entsprechenden Periode $T_k$ für eine Kreisbahn mit Radius $R_0$ ausrechen, wobei $F_g = F_{rad}$ benutzt wird:
 	\begin{equation}\label{polygrav:tc}
 		\frac{GmM_n}{R_0^2}  = m R\underbrace{\left(\frac{2\pi}{T_k}\right)^2}_{=\omega ^2} \Rightarrow T_k = 2\pi\sqrt{\frac{R_0^3}{GM_n}}.
 	\end{equation}
 	Mit dem dritten Keplerschen Gesetz folgt dann
 	\begin{equation}
 	\boxed{	\left(\frac{T_e}{T_k}\right)^2 = \left(\frac{\nicefrac{R_0}{2}}{R_0}\right)^3 \Rightarrow T_e = \pi \sqrt{\frac{R_0^3}{8GM_n}}.}
 	\end{equation}
\end{Answer}
