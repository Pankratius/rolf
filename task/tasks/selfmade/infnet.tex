
\begin{Exercise}[label = infnet, title = Widerstandsnetzwerk, origin = P.Gnädig, difficulty = 4]
	Alle Ecken in dem unendlich großen Widerstandsnetz (siehe Abbildung) haben den Widerstand $R$. Wie groß ist der Widerstand zwischen $A$ und $B$?
\end{Exercise}

\begin{figure}[h]
	\centering
	\begin{tikzpicture}
		\draw
		(-2,1) to[/tikz/circuitikz/bipoles/length=20pt, R] (0,1)
		(0,1) to[/tikz/circuitikz/bipoles/length=20pt, R](1,1)
		(1,1) to[/tikz/circuitikz/bipoles/length=20pt, R](3,1)
		(-1.75,1)to[/tikz/circuitikz/bipoles/length=20pt, R](-1.75,0)
		(-1.75,0)to[/tikz/circuitikz/bipoles/length=20pt, R](-1.75,-1)
				(-2,0) to[/tikz/circuitikz/bipoles/length=20pt, R] (0,0)
				(0,0) to[/tikz/circuitikz/bipoles/length=20pt, R](1,0)
				(1,0) to[/tikz/circuitikz/bipoles/length=20pt, R](3,0)
						(-2,-1) to[/tikz/circuitikz/bipoles/length=20pt, R] (0,-1)
						(0,-1) to[/tikz/circuitikz/bipoles/length=20pt, R](1,-1)
						(1,-1) to[/tikz/circuitikz/bipoles/length=20pt, R](3,-1)
				(-.25,1)to[/tikz/circuitikz/bipoles/length=20pt, R](-.25,0)
				(-.25,0)to[/tikz/circuitikz/bipoles/length=20pt, R](-.25,-1)
					(1.25,1)to[/tikz/circuitikz/bipoles/length=20pt, R](1.25,0)
					(1.25,0)to[/tikz/circuitikz/bipoles/length=20pt, R](1.25,-1)
					(2.75,1)to[/tikz/circuitikz/bipoles/length=20pt, R](2.75,0)
					(2.75,0)to[/tikz/circuitikz/bipoles/length=20pt, R](2.75,-1)
			
					;
		\draw (2.75,1) -- (2.75,1.2);
		\draw (-1.75,1) -- (-1.75,1.2);
		\draw (2.75,-1) -- (2.75,-1.2);
		\draw (-1.75,-1)--(-1.75,-1.2);
		\filldraw[black] (-.25,0) circle (1.5pt);
		\filldraw[black] (1.25,0) circle (1.5pt);
		\node at (0,-.2) {$A$};
		\node at (1.5,-.2) {$B$};
		\node at (0.5,0.25){$R$};
		\node at (-2.2,1) {$...$};
			\node at (-2.2,0) {$...$};
				\node at (-2.2,-1) {$...$};
		\node at (3.2,1) {$...$};
			\node at (3.2,0) {$...$};
			\node at (3.2,-1) {$...$};
		\node at (-1.75,-1.5) {$...$};
		\node at (-0.25,-1.5) {$...$};
		\node at (1.25,-1.5) {$...$};
		\node at (2.75,-1.5) {$...$};
				\node at (-1.75,1.5) {$...$};
				\node at (-0.25,1.5) {$...$};
				\node at (1.25,1.5) {$...$};
				\node at (2.75,1.5) {$...$};
		
		

	\end{tikzpicture}
\end{figure}