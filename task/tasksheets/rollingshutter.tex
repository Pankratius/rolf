\documentclass[a4paper,11pt]{article}
\usepackage{exercise}
%um nur aufgaben zu zeigen
%\usepackage[noanswer]{exercise} 
\usepackage{../images/preamble}
\usepackage{rotating}
\usetikzlibrary{decorations.pathmorphing}
\usetikzlibrary{decorations.markings}
\usetikzlibrary{arrows}
\usetikzlibrary{shapes.geometric}
\newcommand{\midarrow}{\tikz \draw[-triangle 90] (0,0) -- +(.02,0);}
\usepackage{xcolor}
%\usepackage{draftwatermark}
%\SetWatermarkText{\textsc{Draft 2}}
%\SetWatermarkScale{3}
%\SetWatermarkColor{red!30}


\usepackage[printwatermark]{xwatermark}
%\newsavebox\mybox
%\savebox\mybox{\tikz[color=red,opacity=0.3]\node{\textsc{Entwurf}};}
%\newwatermark*[
%allpages,
%angle=45,
%scale=10,
%xpos=-4cm,
%ypos=4cm
%]{\usebox\mybox}


\begin{document}
	\vspace*{-2cm}
	\pagestyle{empty}
\begin{framed}
	\noindent
	\scriptsize
	Diese Aufgabe dient als Vorbereitungsaufgabe für das Landesseminar Physik Sachsen/Thüringen 2018.\\
	Die Lösung kann entweder per Mai an 
	 \href{mailto:aaronwild@uni-bonn.de}{aaron.wild@uni-bonn.de} oder bei der Anreise zum Landesseminar (19.01.2018) abgebeben werden, und dort korrigiert und besprochen.
\end{framed}

\noindent

\documentclass[a4paper,11pt]{article}
\usepackage{exercise}
%um nur aufgaben zu zeigen
%\usepackage[noanswer]{exercise} 
\usepackage{../images/preamble}
\usepackage{rotating}
\usetikzlibrary{decorations.pathmorphing}
\usetikzlibrary{decorations.markings}
\usetikzlibrary{arrows}
\usetikzlibrary{shapes.geometric}
\newcommand{\midarrow}{\tikz \draw[-triangle 90] (0,0) -- +(.02,0);}
\usepackage{xcolor}
%\usepackage{draftwatermark}
%\SetWatermarkText{\textsc{Draft 2}}
%\SetWatermarkScale{3}
%\SetWatermarkColor{red!30}


\usepackage[printwatermark]{xwatermark}
%\newsavebox\mybox
%\savebox\mybox{\tikz[color=red,opacity=0.3]\node{\textsc{Entwurf}};}
%\newwatermark*[
%allpages,
%angle=45,
%scale=10,
%xpos=-4cm,
%ypos=4cm
%]{\usebox\mybox}


\begin{document}
	\vspace*{-2cm}
	\pagestyle{empty}
\begin{framed}
	\noindent
	\scriptsize
	Diese Aufgabe dient als Vorbereitungsaufgabe für das Landesseminar Physik Sachsen/Thüringen 2018.\\
	Die Lösung kann entweder per Mai an 
	 \href{mailto:aaronwild@uni-bonn.de}{aaron.wild@uni-bonn.de} oder bei der Anreise zum Landesseminar (19.01.2018) abgebeben werden, und dort korrigiert und besprochen.
\end{framed}

\noindent

\documentclass[a4paper,11pt]{article}
\usepackage{exercise}
%um nur aufgaben zu zeigen
%\usepackage[noanswer]{exercise} 
\usepackage{../images/preamble}
\usepackage{rotating}
\usetikzlibrary{decorations.pathmorphing}
\usetikzlibrary{decorations.markings}
\usetikzlibrary{arrows}
\usetikzlibrary{shapes.geometric}
\newcommand{\midarrow}{\tikz \draw[-triangle 90] (0,0) -- +(.02,0);}
\usepackage{xcolor}
%\usepackage{draftwatermark}
%\SetWatermarkText{\textsc{Draft 2}}
%\SetWatermarkScale{3}
%\SetWatermarkColor{red!30}


\usepackage[printwatermark]{xwatermark}
%\newsavebox\mybox
%\savebox\mybox{\tikz[color=red,opacity=0.3]\node{\textsc{Entwurf}};}
%\newwatermark*[
%allpages,
%angle=45,
%scale=10,
%xpos=-4cm,
%ypos=4cm
%]{\usebox\mybox}


\begin{document}
	\vspace*{-2cm}
	\pagestyle{empty}
\begin{framed}
	\noindent
	\scriptsize
	Diese Aufgabe dient als Vorbereitungsaufgabe für das Landesseminar Physik Sachsen/Thüringen 2018.\\
	Die Lösung kann entweder per Mai an 
	 \href{mailto:aaronwild@uni-bonn.de}{aaron.wild@uni-bonn.de} oder bei der Anreise zum Landesseminar (19.01.2018) abgebeben werden, und dort korrigiert und besprochen.
\end{framed}

\noindent

\documentclass[a4paper,11pt]{article}
\usepackage{exercise}
%um nur aufgaben zu zeigen
%\usepackage[noanswer]{exercise} 
\usepackage{../images/preamble}
\usepackage{rotating}
\usetikzlibrary{decorations.pathmorphing}
\usetikzlibrary{decorations.markings}
\usetikzlibrary{arrows}
\usetikzlibrary{shapes.geometric}
\newcommand{\midarrow}{\tikz \draw[-triangle 90] (0,0) -- +(.02,0);}
\usepackage{xcolor}
%\usepackage{draftwatermark}
%\SetWatermarkText{\textsc{Draft 2}}
%\SetWatermarkScale{3}
%\SetWatermarkColor{red!30}


\usepackage[printwatermark]{xwatermark}
%\newsavebox\mybox
%\savebox\mybox{\tikz[color=red,opacity=0.3]\node{\textsc{Entwurf}};}
%\newwatermark*[
%allpages,
%angle=45,
%scale=10,
%xpos=-4cm,
%ypos=4cm
%]{\usebox\mybox}


\begin{document}
	\vspace*{-2cm}
	\pagestyle{empty}
\begin{framed}
	\noindent
	\scriptsize
	Diese Aufgabe dient als Vorbereitungsaufgabe für das Landesseminar Physik Sachsen/Thüringen 2018.\\
	Die Lösung kann entweder per Mai an 
	 \href{mailto:aaronwild@uni-bonn.de}{aaron.wild@uni-bonn.de} oder bei der Anreise zum Landesseminar (19.01.2018) abgebeben werden, und dort korrigiert und besprochen.
\end{framed}

\noindent

\input{../tasks/kalda/rollingshutter}



\end{document}



\end{document}



\end{document}



\end{document}