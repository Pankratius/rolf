\documentclass[a4paper]{article}
\usepackage[utf8]{inputenc}
\usepackage[german]{babel}
\usepackage{amsmath}
\usepackage{amsthm}
\usepackage{amsfonts}
\usepackage{amssymb}
\usepackage{graphicx}
\usepackage{wrapfig}
\usepackage{float}
\usepackage{nicefrac}
\usepackage{tkz-euclide}
\usetkzobj{all}
\usepackage{tikz}
\usepackage{exercise}
\usepackage{import}
\usepackage{hyperref}
\usepackage[european]{circuitikz}
\usepackage{rotating}

\usepackage{framed}
\usepackage{setspace} %Für \setstretch{}, welches Zeilenabstände reguliert.

\usepackage{caption}
\usepackage{subcaption}


\usepackage[left=2cm,right=2.5cm,top=2.5cm,bottom=2cm]{geometry}

\usepackage{titlesec}
\numberwithin{equation}{Exercise}


  \renewcommand{\ExerciseHeader}{\flushleft\textsf{\textbf{
               Aufgabe \ExerciseHeaderNB} (\ExerciseHeaderTitle)\smallskip\newline
               }}
%\ExerciseHeaderDifficulty
 \renewcommand{\AnswerHeader}{\flushright \textbf{\textsf{Lösung
              \ExerciseHeaderNB} }\ExerciseHeaderOrigin\smallskip\newline}

\renewcommand{\ExerciseHeaderTitle}{\ExerciseTitle}
 \setlength{\ExerciseSkipBefore}{0\baselineskip}


\begin{document}
	\vspace*{-2cm}
	\parbox{4cm}{\includegraphics[width=2.5cm]{../images/ROLF4.png}}
	\parbox{10.6cm}{\setstretch{2.0} \centering{ \huge \textsf{Aufgabenserie 2}}\\ Abgabe: 19. Juni \\ \vspace*{-.5cm} }
	
	

\thispagestyle{empty}
\begin{framed}
	\noindent
	\scriptsize
	Die Aufgaben sollten bis zum \textbf{19. Juni} bearbeitet werden. Die Lösungen schickt ihr entweder an \href{mailto:physikrolf@gmail.com}{physikrolf@gmail.com}, oder per Post an: Domino's Pizza, c/o Aaron Wild, Rollgasse 13, 99423 Weimar.
	Jede Aufgabe hat eine bestimmte Anzahl an erreichbaren Punkten. Wie viele das sind, müsst ihr raten. Versucht, die Lösungen so genau wie möglich aufzuschreiben. Für besonders schnelle/gute/witzige Lösungen kann es Bonuspunkte geben.
\end{framed}

\noindent

\begin{Exercise}[label = throwim1, difficulty = 3, origin = Aaron Wild, title = Schräger Wurf]
	Das folgende Bild zeigt eine Aufnahme von einem schrägen Wurf. Dabei ist $S$ der Abwurfpunkt. Die roten Punkte geben die Position des Körpers 2 bzw. 4 Sekunden nach dem Abwurf an. Die blauen Punkte zeigen die Position des Körpers kurz nachdem die roten Punkte geblinkt haben.\\
	Bestimme die Richtung der Gravitationsbeschleunigung, die Abwurfgeschwindigkeit und den Abwurfwinkel relativ zum Boden.
\end{Exercise}
\begin{figure}[h]
	\centering
	\input{../tasks/selfmade/throwim2}
\end{figure}

\begin{Exercise}[label = hypercube, origin = Aaron Wild, difficulty = 5, title =  Widerstandswürfel]
Ein $n$-dimensionaler Hyperwürfel ist die Verallgemeinerung eines Würfels auf $n$ Dimensionen. Seine Konstruktion kann man sich so vorstellen, das ein $n-1$-dimensionaler Hyperwürfel im $n$-dimensionalen Raum parallelverschoben wird, und man das daraus entstandene Volumen betrachtet.\\
Ein solcher $n$-dimensionaler Hyperwürfel hat $2^n$ Eckpunkte und $n2^{n-1}$ Seitenkanten.\\
Wir betrachten nun einen $n$-dimensionalen Hyperwürfel ($n\geq 1$), bei dem alle Seitenkanten einen Widerstand von $r$ haben. Zeige, dass der Widerstand zwischen zwei benachbarten Eckpunkten 
\begin{equation}
	R = \frac{2-2^{1-n}}{n}r = \frac{2^n-1}{n2^{n-1}}r
\end{equation}
beträgt. Überlege dir an einem $n$ deiner Wahl, dass das Ergebnis dort sinnvoll ist.
\end{Exercise}
\begin{Exercise}[label = cws, origin = Est.-Fin.-Auswahlklausur, title = Kochen mit der Sonne, difficulty = 4]
	Sonnenlicht wird durch eine Linse mit dem Durchmesser $d = 5~\mathrm{cm}$ auf eine schwarze, dünne Platte fokusiert, hinter der sich ein Spiegel befindet. Die Brennweite der Linse beträgt$f = 10~\mathrm{cm}$. Das Verhältnis von Sonnendurchmesser zu Sonnenentfernung ist etwa $\alpha=9.206\cdot10^{-3}$ und die Intensität des Sonnenlichts auf der Erde beträgt etwa $1300~\mathrm{\nicefrac{W}{m^2}}$.\\
	Berechne die Temperatur des beleuchteten Punkts.
 \end{Exercise}





\newpage
\begin{turn}{270}
\definecolor{qqttcc}{rgb}{0.,0.2,0.8}
\definecolor{ffqqqq}{rgb}{1.,0.,0.}
\definecolor{ffqqtt}{rgb}{1.,0.,0.2}
\definecolor{cqcqcq}{rgb}{0.7529411764705882,0.7529411764705882,0.7529411764705882}
\begin{tikzpicture}[line cap=round,line join=round,>=triangle 45,x=0.75cm,y=0.75cm]
\draw [color=cqcqcq,, xstep=1.5cm,ystep=1.5cm] (-8.5,0.6) grid (23.,20.);
\clip(-8.5,0.6) rectangle (23.,20.);
\draw (-5.588239388018447,9.518801182219294) node[anchor=north west] {$S$};
\begin{scriptsize}
\draw [fill=black] (-5.65741495087795,9.439855831290508) circle (2.5pt);
\draw [fill=ffqqtt] (1.9550099267845686,14.114279339778834) circle (2.0pt);
\draw [fill=ffqqqq] (17.86598524604952,10.425615052655644) circle (2.0pt);
\draw [fill=qqttcc] (2.6268608640983873,14.364446314147006) circle (2.0pt);
\draw [fill=qqttcc] (18.279983363360785,9.990865906507295) circle (2.0pt);
\end{scriptsize}
\end{tikzpicture}
\end{turn}



\end{document}
