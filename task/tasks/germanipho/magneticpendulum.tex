\begin{Exercise}[label = magneticpendulum, origin = {3. Runde zur 42. IPhO 2011}, title = Geladene Kugel am Faden, difficulty = 2]
	Eine geladene, kleine Kugel der Masse $m = 10~\mathrm{g}$ hängt an einme isolierenden, masselosen Faden der Länge $\ell = 1~\mathrm{m}$ von einer Decke herab.\\
	Sie befindet sich in einem homogenen, senkrechten Magnetfeld der Feldstärke $B = 50~\mathrm{mT}$. Die Kugel wird so in eine horizontale Rotation versetzt, dass der Faden einen Winkel von $\alpha = 30^\circ$ mit der Vertikalen einschließt.\\
	Die Rotationsfrequenzen im bzw. gegen den Uhrzeigersinn unterscheiden sich um $\Delta f = 2.0~\cdot 10^{-3}~\mathrm{Hz}$.
	\Question Wie groß ist die Ladung $Q$ der Kugel.
	\Question Wie groß ist der Mittel $\overline{f}$ der Rotationsfrequenzen?
\end{Exercise}