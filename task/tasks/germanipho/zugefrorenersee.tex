\begin{Exercise}[label = Zugefrorener See, origin = {3. Runde zur 43. IPhO 2012}, title = Zugefrorener See, difficulty = 3]
Ein See ist von einer Eisschicht der Dicke $d$ bedeckt und die Umgebungstemperatur beträgt seit langer Zeit $T$.
\begin{enumerate}
\item Schätze die Rate ab, mit der die Dicke der Eisschicht zunimmt. Drücke das Ergebnis in Abhängigkeit von Schmelzwärmen, Leitfähigkeiten, spezifischen Wärmekapazitäten und Dichten aus.
\item Wie lange dauert es, bis sich die Dicke der Schicht verdreifacht hat? 
\end{enumerate}    	
\end{Exercise}
