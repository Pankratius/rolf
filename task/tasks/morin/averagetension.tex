\begin{Exercise}[title = Seilkraft, origin = {Morin - Classical Mechanics}, difficulty = 3, label = pucks]
Gegeben sei ein Pendel der Länge $\ell$ mit maximalem Auslenkungswinkel $A$, wobei $A \ll 1$. Für den Auslenkungswinkel $\theta(t)$ und die Winkelgeschwindigkeit $\omega(t)$ gilt 
\begin{subequations}\label{averagetension:simpleharmonic}
	\begin{equation}\label{averagetension:angle}
		\theta(t) = A\cos\left(\omega_0t\right)
	\end{equation}
	\begin{equation}\label{averagtension:angularvel}
		\omega(t) = A\omega_0\sin\left(\omega_0t\right),
	\end{equation}
\end{subequations}
wobei $\omega_0 = \sqrt{g/\ell}$ die Kreisfrequenz der Schwingung ist.\\
Bestimme näherungsweise (besser als $mg$!) die Spannkraft in der Pendelschnur.
\end{Exercise}