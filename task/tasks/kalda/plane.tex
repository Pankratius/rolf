\begin{Exercise}[title = Flugzeuge, origin = Jaan Kalda]
	Zwei Flugzeuge fliegen auf gleicher Höhe mit den Geschwindigkeiten $v_1 = 600~\mathrm{\frac{km}{h}}$ und $v_2 = 800~\mathrm{\frac{km}{h}}$. Dabei befinden sie sich am Anfang auf den Eckpunkten eines gleichschenklig-rechtwinkligen Dreiecks mit der Seitenlänge $a = 20~\mathrm{km}$. Berechne den geringsten Abstand zwischen den beiden Flugzeugen unter der Annahme konstanter Geschwindigkeit.
\end{Exercise}

\begin{tikzpicture}
	\draw[dashed] (-2,0) -- (.5,0);
	\draw[->,thick] (-2.2,0) to node[midway, above] {$v_1$} (-2,0);
	\draw[<->] (-2,.5) to node[midway, above]{$a$} (0,.5);
	
	\draw
	
\end{tikzpicture}