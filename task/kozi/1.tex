\documentclass[a4paper]{article}
\usepackage{exercise}
%um nur aufgaben zu zeigen
%\usepackage[noanswer]{exercise} 
\usepackage{../images/preamble}

\pagestyle{fancy}
\fancyhead[L]{\includegraphics[width=2cm]{../images/logo_scaled.pdf}}
\fancyhead[R]{\textsc{Lösung Aufgabenserie 1}}
\begin{document}
%	\vspace*{-2cm}
%	\parbox{4cm}{\includegraphics[width=2.5cm]{../images/ROLF4.png}}
%	\parbox{10.6cm}{\setstretch{2.0} \centering{ \huge \textsf{Aufgabenserie 1}}\\ Abgabe: 19. Mai \\ \vspace*{-.5cm} }
	
\vspace*{-1cm}
\parbox{4cm}{\vspace{-0.2cm}\includegraphics[width=5cm]{../images/logo_scaled.pdf}}
\parbox{10.6cm}{\setstretch{2.0} \centering{ \huge \textsf{Aufgabenserie 1 
			}}\\
			Abgabe: 19. Mai 2017 \\ 
			\vspace*{-.5cm} }
		\vspace{0.5cm}
	

\thispagestyle{empty}
\begin{framed}
	\noindent
	\scriptsize
	Die Aufgaben sollten bis zum \textbf{19. Mai} bearbeitet werden. Die Lösungen schickt ihr an \href{mailto:physikrolf@gmail.com}{physikrolf@gmail.com}.
	Jede Aufgabe hat eine bestimmte Anzahl an erreichbaren Punkten. Wie viele das sind, müsst ihr raten. Versucht, die Lösungen so genau wie möglich aufzuschreiben. Für besonders schnelle/gute/witzige Lösungen kann es Bonuspunkte geben.\\ Die aktuellen Aufgaben sowie alle alten Aufgabenserien mit Lösungen findet ihr auch auf \url{pankratius.github.io/rolf}

\end{framed}

\noindent

	\begin{Exercise}[label=ipho201741, title = Eisenbahnwaggon,difficulty={2},origin = {4. Runde IPhO 2017, Kurzaufgabe}]
 Ein Eisenbahnwaggon der Masse $m_1 = 37~\mathrm{t}$ rollt reibungsfrei auf einem Bahngleis und stößt mit einem anderen Waggon zusammen. Beide verkuppeln sich und rollen zusammen weiter. Dabei geben sie 35\% der anfangs vorhandenen kinetischen Energie als Wärme ab. Wie groß ist die Masse $m_2$ des zweiten Waggons?
	\end{Exercise}
\begin{Answer}[ref = ipho201741]
	Da keine äußeren Kräfte wirken, bleibt der Impuls erhalten.\\ Gleichzeitig bleibt die Gesamtenergie erhalten,
	\begin{equation}\label{energycons}
		E_{kin,v} = E_{kin,n}+E_{w}.
	\end{equation}
	Hierbei ist $E_{kin,v}$ die kinetische Energie vor dem Stoß, $E_{kin,n}$ die kinetische Energie nach dem Stoß, und $E_{w}$ die abgegebene Wärmeenergie. Es ist gegeben, dass $E_{w} = \eta \cdot E_{kin,v}$ gilt, wobei $\eta = 0.35$ gilt. Um die Impulserhaltung zu benutzten kann man entweder die kinetische Energie durch die Impulse oder aber durch die Geschwindigkeiten ausdrücken.\\
	Drückt man die kinetische Energie durch die Impulse aus, so ist im Allgemeinen $E_{kin} = \frac{p^2}{2m}$. Da der Impulserhalten bleibt, gilt $p:=p_v = p_n = \mathrm{konst.}$. Damit lässt sich die Energieerhaltung schreiben als
	\begin{equation}\label{penenergycons}
		\frac{p^2}{2m_1}= \frac{p^2}{2\left(m_1+m_2\right)} + \eta\cdot \frac{p^2}{2m_1},
	\end{equation}
	da nach dem Stoß nur noch ein \glqq Waggon\grqq{} der Masse $m_1+m_2$ da ist.
	Division durch $p^2$ auf beiden Seiten führt auf
	\begin{equation}\label{masscond}
		\frac{1}{2m_1}= \frac{1}{2\left(m_1+m_2\right)}+\frac{\eta}{2m_1}.
	\end{equation}
	Diese Gleichung erhält nur noch die bekannten Größen $m_1$ und $\eta$, sowie die unbekannte Größe $m_2$. Subtrahieren von $\frac{\eta}{2m_1}$, multiplizieren mit 2 und anschließendes Bilden des Reziproken führt auf
	\begin{equation*}\label{solp}
	\boxed{
		\frac{1}{m_1}\left(1-\eta\right) = \frac{1}{m_1+m_2} \Rightarrow m_2 = m_1\left(\frac{1}{1-\eta}-1\right)\approx 20~\mathrm{t}.}
	\end{equation*}
	Das gleiche kann man auch mit den Geschwindigkeiten machen. Hier hat die Energieerhaltung die Form
	\begin{equation}\label{venergycons}
		m_1 \frac{v_v^2}{2} = \left(m_1+m_2\right)\frac{v_n^2}{2} + m_1\eta  \cdot \frac{v_v^2}{2}.
	\end{equation}
	Explizit ausgeschrieben besagt die Impulserhaltung, dass 
	\begin{equation}\label{secondvel}
		m_1 v_v = \left(m_1+m_2\right)v_n \Rightarrow v_n = \frac{m_1}{m_1+m_2}v_v
	\end{equation}
	gilt. Einsetzen von \eqref{secondvel} in \eqref{venergycons} führt auf
	\begin{equation*}
	m_1\frac{v_v^2}{2} = \left(m_1+m_2\right)\frac{v_v^2 m_1^2}{2\left(m_1+m_2\right)^2} + m_1\eta \cdot \frac{v_v^2}{2}.
	\end{equation*}
	Dividieren durch $m_1^2v_v^2$ führt auf 
	\begin{equation}\label{vmasscond}
		\frac{1}{2m_1} = \frac{1}{2\left(m_1+m_2\right)} + \frac{\eta}{2m_1}.
	\end{equation}
	Die Bedingung, die die Massen erfüllen müssen, wenn wir Geschwindigkeiten betrachten (Gleichung \eqref{vmasscond}) entspricht also genau der, die erfüllt sein muss, wenn wir Impulse betrachten (Gleichung \eqref{masscond}). Das sollte auch so sein.
\end{Answer}

\begin{Exercise}[label = ipho20170402, title = Zwei Linsenpositionen , difficulty = 4, origin = 4. Runde IPhO 2017 ]
Versucht man, einen Gegenstand mit einer Sammellinse auf eine im festen Abstand $a$ positionierte Fläche scharf abzubilden, stellt man fest, dass das für zwei unterschiedliche Linsenpositionen möglich ist.\\
Betrachte einen Aufbau, bei dem $a= 120~\mathrm{m}$ gilt. Das Verhältnis der Bildgrößen beträgt $\frac{1}{9}$. Wie groß ist die Brennweite der verwendeten Linse? 
\end{Exercise}	
\begin{Answer}[ref = ipho20170402]
	Es entsteht einmal ein kleines Bild, und einmal ein großes Bild. Alle Variabeln, die sich auf das kleine Bild beziehen, haben einen Index 1, und alle, die sich auf das große Bild beziehen, einen Index 2.\\
	Das in der Aufgabenstellung definierte Verhältnis definieren wir als $\varepsilon^2$. Dann gilt
	\begin{equation}\label{bessel:epsdef}
		\varepsilon^2=\frac{B_1}{B_2} = \frac{1}{9},
	\end{equation}
	wobei $B$ jeweils die Bildgröße bezeichnet. Aus der Strahlenoptik wissen wir, dass für die Gegenstandsgröße $G$, die Bildgröße $B$, die Gegenstandsweite $g$ und die Bildweite $b$ im Allgemeinen gilt
	\begin{equation*}
		\frac{B}{G} = \frac{b}{g} \Rightarrow B = \frac{b}{g}\cdot G \Rightarrow B_1 = \frac{b_1}{g_1}\cdot G~\mathrm{und} B_2 ~= \frac{b_2}{g_2}\cdot G.
	\end{equation*}
	Diese Ausdrücke für $B_1$ und $B_2$ kann man jetzt in \eqref{bessel:epsdef} einsetzen. Da die Gegenstandsgröße für beide Bilder gleich groß ist (es wird ja immer noch der  gleiche Gegenstand abgebildet), kürzt sich $G$, sodass wir auf
	\begin{equation}\label{bessel:masss}
		\varepsilon^2 = \frac{b_1}{g_1}\cdot G \cdot \frac{g_2}{b_2}\cdot \frac{1}{G} = \frac{b_1 g_2}{b_2 g_1} 
	\end{equation}	
	kommen.\\
	Diese Gleichung kann einfacherer werden, wenn man sich überlegt, wie $b_1$ und $g_2$ bzw. $b_2$ und $g_1$ im Zusammenhang stehen. Dafür kann man sich vorstellen, dass man geraden den Aufbau hat, der die scharfe Abbildung des großen Bildes liefert. Tauscht man jetzt Linse und Schirm aus, muss (weil der Strahlengang umkehrbar ist) wieder ein scharfes Bild entstehen. Das ist jetzt also das Kleine. Was man aber nur gemacht hat, ist Bildweite und Gegenstandsweite zu tauschen. Also ist $b_1 = g_2$ und $b_2 = g_1$. \\
	Setzten wir diese beiden Ausdrücke in \eqref{bessel:masss} ein, so kommen wir 
	\begin{equation}\label{bessel:simple}
		\varepsilon^2= \frac{b_1^2}{g_1^2} \Rightarrow b_1 = \varepsilon g_1.
	\end{equation}
	Es gibt jetzt mehrere Möglichkeiten, wie man mit diesem Ergebnis weiter rechnet.\\
	Eine davon ist, die sog. Besselgleichung zu nehmen. Diese führt neben dem Abstand von Gegenstand und Bild $a$ auch noch den Abstand $e$ zwischen den beiden Linsenpositionenen, bei denen die Abbildung scharf ist, ein. In unserem Fall ist das also $e = g_1 - g_2$\footnote{Das Vorzeichen von $e$ ist nicht wichtig, weil nur $e^2$ in \eqref{bessel:bessel} eingeht.}. Damit kann man die Brennweite $f$ mit der Gleichung
	\begin{equation}\label{bessel:bessel}
		f = \frac{a^2-e^2}{4a}
	\end{equation}
	ausrechnen. \\
	Da die Linse dünn ist, muss $a = b_1 + g_1$ sein. Also lässt sich die Gegenstandsweite $g_1$ schreiben als \eqref{bessel:simple}
	\begin{equation}\label{bessel:gw}
		a = g_1 + b_1 = g_1 + \varepsilon g_1 = g_1\left(1+\epsilon\right) \Rightarrow g_1 = \frac{a}{1+\varepsilon}.
	\end{equation} 
	Wir können nun \eqref{bessel:simple} und \eqref{bessel:bessel} nutzten, um $e$ nur durch $a$ und $\varepsilon$ auszudrücken. Zuerst ist $g_2 = b_1$, also $e = g_1 - b_1$. Einfach einsetzen ergibt
	\begin{equation}
		e = g_1 - b_1 = \overset{\eqref{bessel:simple}}{=} g_1 - \varepsilon g_1 = g_1\left(1-\varepsilon\right) \overset{\eqref{bessel:gw}}{=} a \frac{1-\varepsilon}{1+\varepsilon}.
	\end{equation}
	Diesen Ausdruck können wir nun in \eqref{bessel:bessel} einsetzten
	\begin{equation}\label{bessel:fbessel}
	\boxed{
		f = \frac{a^2-e^2}{4a} = a\cdot \frac{1-\left(\nicefrac{e}{a}\right)^2}{4} = a\cdot \frac{1-\left(\frac{1-\varepsilon}{1+\varepsilon}\right)^2}{4} =a\cdot  \frac{\varepsilon}{\left(1+\varepsilon\right)^2}.}
	\end{equation}
	Mit $\varepsilon =\sqrt{\nicefrac{1}{9}} =  \nicefrac{1}{3}$ ist $f = 22.5~\mathrm{m}$.\\
	Das gleiche Ergebnis kann man auch mit der Abbildungsgleichung finden. Wir wissen, dass 
	\begin{equation}\
		\frac{1}{f} = \frac{1}{g_1} + \frac{1}{b_1} \Rightarrow f = \frac{g_1b_1}{g_1+b_1}
	\end{equation}
	gilt. Wir können jetzt wieder die Ausdrücke für $b_1$ und $g_1$ aus \eqref{bessel:simple}  und \eqref{bessel:gw} einsetzen
	\begin{equation}\boxed{
		f = \frac{g_1 b_1}{g_1+b_1} \overset{\eqref{bessel:simple}}{=}  \frac{g_1^2\varepsilon}{g_1+\varepsilon g_1} \overset{\eqref{bessel:gw}}{=} a^2\cdot \frac{\varepsilon\frac{1}{\left(1+\varepsilon\right)^2}}{\frac{a}{1+\varepsilon}+\varepsilon \cdot \frac{a}{1+\varepsilon}} = a\cdot \frac{\varepsilon}{\left(1+\varepsilon\right)^2}.}
	\end{equation}
	Es kommt also mit beiden Methoden genau das gleiche raus. Ach was!
	

\end{Answer}
\begin{Exercise}[label = ampel, title = Gute Ampel, difficulty = 2, origin = Alte IPhO-Aufgabe ]
	Ein Auto nähert sich auf einer Straße einer Ampel mit der Geschwindigkeit $v_0$.  Der Gleitreibungskoeffizient zwischen Straße und Auto beträgt $\mu$.\\ Die Ampel springt von Grün auf Gelb, wenn der Abstand Auto-Ampel gerade $s_0$ beträgt. Wie lang muss die Ampel gelb sein, damit der Autofahrer, unabhängig von $v_0$, entweder vor der Ampel zum Stehen kommt, oder aber mit konstanter Geschwindigkeit die Ampel kreuzt?
\end{Exercise}	
\begin{Answer}[ref = ampel]
	Diese Aufgabe kann man über mehrere Wege lösen.\\
	 Am Einfachsten geht es mit der Energieerhaltung. Wenn das Auto nach zurücklegen der Strecke $s_0$ auf jeden Fall zum Stehen gekommen sein, muss die auf dieser Strecke aufgebrachte Reibungsarbeit $E_r$ mindestens so groß sein, wie die kinetische Energie $E_{kin}$ vor dem Bremsen. Damit gillt
	 \begin{equation}\label{energycond}
	 E_{r} \geq E_{kin} \Rightarrow mg\mu s_0 \geq \frac{mv_0^2}{2}.
	 \end{equation}
	 Damit man vor der Ampel bei gegebener Strecke stehen bleiben kann, muss die Geschwindigkeit so gewählt werden, dass \eqref{energycond} erfüllt ist. Durch das Wurzelziehen ändert sich die Richtung der Ungleichung nicht, sodass man höchsten mit einer Geschwindigkeit von 
	 \begin{equation}\label{vfriccond}
	 	v_0\geq \sqrt{2s_0g\mu}
	 \end{equation}
	 vor der Ampel stehen bleiben kann.\\
	 Wenn man die Ampel bei gegebener Gelbphase ohne zu Bremsen kreuzen will, muss die Geschwindigkeit $v_0$ dafür groß genug sein, also
	 \begin{equation}\label{velcond}
	 	v_0 \geq  \frac{s_0}{t_g}
	 \end{equation}
	Da ein Bremsen aber immer möglich seien soll, können wir beide Bedingungen zusammen in einer Ungleichung schreiben
	\begin{equation}\label{stvcond}
		\sqrt{2s_0g\mu}\geq v_0 \geq \frac{s_0}{t}.
	\end{equation}
	Nach Aufgabenstellung soll die Zeit der Gelbphase $t_g$ so gewählt werden, dass die Bedingung \eqref{stvcond} unabhängig von der Geschwindigkeit ist. Das ist der Fall, wenn
	\begin{equation}\label{energytcond}
	\boxed{
	\sqrt{2s_0g\mu} \leq \sqrt{\frac{s_0}{2g\mu}}}
	\end{equation}
	gilt.
\end{Answer}





\end{document}
