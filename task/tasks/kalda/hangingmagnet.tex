\begin{Exercise}[difficulty = 3, origin = Estnisch-Finnische Physikolympiade 2014, title = Kraft zwischen Magneten, label = hangingmagnet]
	Um die Kraft zwischen zwei kleinen Stabmagneten zu finden, wird folgendes Experiment durchgeführt:\\
	Einer der beiden Magneten wird mit einem (masselosen) Faden der Länge $\ell = 1~\mathrm{m}$ an der Decke befestigt.\\
	Der zweite wird langsam an den ersten geführt, sodass die horizontalen Symmetrieachsen der beiden Magneten immer auf einer Gerade liegen.\\
	Als die Distanz der beiden Magneten gerade $4~\mathrm{cm}$ beträgt, hat sich der hängende Magnet $1~\mathrm{cm}$ bewegt. In diesem Moment verbinden sich die beiden Magneten schlagartig zueinander.\\
	Es kann angenommen werden, dass die Kraft $\vec{F}_m$ zwischen den beiden Magneten in der Form $|\vec{F}_m|\propto d^{-n}$ modelliert werden kann, wobei $d$ der Abstand der beiden Magneten ist, und $n\in \mathbb{N}$.\\
	Wie groß ist $n$?
\end{Exercise}