\begin{Exercise}[label = glacier, origin = {Orpheus-Seminar 2014, Lucas Rettenmeier}, difficulty = 3, title = Gletscherschmelze]
Wenn sich ein Gletscher zu schnell über seinen Untergrund bewegt, kann er aufgrund von Reibung zu schmelzen beginnen. \\
Um diesen Vorgang besser zu verstehen, betrachten wir eine quaderförmige Eisplatte mit Seitenlängen $a$ und Höhe $H$, die mit konstanter Geschwindigkeit über eine horizontale Ebene bewegt wird. Der Gleitreibungskoeffizient zwischen dem Eis und der Fläche beträgt $\mu$.\\
Nimm an, dass die Oberfläche des Gletschers eine konstante Temperatur $\vartheta_0 <0^\circ$ hat, und der Untergrund keine Wärme leitet. Die Dichte des Eis beträgt $\rho$, und die Wärmeleitfähigkeit $\kappa$. \\
Bei welcher kritischen Geschwindigkeit $v$ beginnt das Eis zu schmelzen? Auf welche Höhe schmilzt der Gletscher, wenn er sich mit dem dreifachen der kritischen Geschwindigkeit bewegt?
\end{Exercise}
\begin{Answer}[ref = glacier]
	Während der Bewegung wirkt die ganze Zeit eine Gleitreibungskraft $F_r$. Diese kann durch den gegebenen Gleitreibungskoeffizienten, die Dichte des Eisbergs und seine Abmessungen ausgerechnet werden,
	\begin{equation}\label{glacier:friction}
		F_r = F_g\mu = mg\mu = \rho V g\mu = \rho a^2Hg\mu.
	\end{equation}
Durch die Reibungsarbeit nimmt der Gletscher Energie auf, die nach einer Strecke $s$ gegeben ist durch $F_rs$. Damit lässt sich die Leistung schreiben als
\begin{equation}\label{glacier:power}
	P = \frac{\Delta E}{\Delta t} = \rho a^2 H g\mu \cdot \frac{\Delta s}{\Delta t} = \rho a^2 H g\mu v,
\end{equation}
wobei $v=\frac{\Delta s}{\Delta t}$ die Gletschergeschwindigkeit ist.\\
Diese Leistung führt nun zur Erwärmung des Gletschers. Dies geschieht durch Wärmeleitung, welche durch \footnote[2]{vgl. Serie 2, Aufgabe 2}
\begin{equation}\label{glacier:powerh}
	P = \frac{A\kappa \left(\vartheta_o-\vartheta_u\right)}{H} =- \frac{a^2\kappa \vartheta_o}{H},
\end{equation}
gegeben ist.
Dabei ist $H$ genau die Dicke der wärmeleitenden Schicht ist, $A =a^2$ die Gletscheroberfläche und $\vartheta_u$ die Temperatur der Gletscherunterseite ist, die beim Schmelzen genau $\vartheta_u = 0^\circ$ beträgt. \\
Weil \eqref{glacier:power} und \eqref{glacier:powerh} den gleichen Vorgang beschreiben, können wir sie einfach gleichsetzen, und nach der Geschwindigkeit $v$ auflösen
\begin{equation}\label{glacier:vel}
	\boxed{ \rho a^2 H g\mu v = - \frac{a^2\kappa \vartheta_o}{H} \Rightarrow v = \frac{-\kappa \vartheta_o}{H^2\rho g\mu }.}
\end{equation} 
Stellen wir diese Gleichung jetzt nach $H$ um, erhalten wir die stabile Gletscherhöhe als Funktion der Geschwindigkeit, $H\left(v\right)$. Durch einfaches einsetzen erhalten wir dann die neue Höhe $H'$,
\begin{equation}
	\boxed{
		H' = H\left(3v\right) = \sqrt{\frac{-\kappa \vartheta_o}{3v\rho g \mu}} = \frac{1}{\sqrt{3}} H = \frac{\sqrt{3}}{3}H.
		}
\end{equation}
\end{Answer}