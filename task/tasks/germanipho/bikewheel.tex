\begin{Exercise}[label = bikewheel, title = Wassertropfen und Fahrrad, origin = Ein Klassiker, difficulty = 3]
	Ein Fahrradreifen hat einen Durchmesser von etwa $70~\mathrm{cm}$. Der äußere Rand bewegt sich mit einer Geschwindigkeit von $35~\mathrm{\nicefrac{m}{s}}$. Auf dem Reifen befinden sich Wassertröpfchen, die durch die Drehung in die Luft geschleudert werden. Bestimme, wie hoch sie dabei maximal fliegen können. Wie groß ist der Winkel, den sie dabei mit der Parallen zum Boden durch den Reifenmittelpunkt einschließen?
\end{Exercise}

	
