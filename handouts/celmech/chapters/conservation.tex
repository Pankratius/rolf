\section{Bewegung im Zentralkraftfeld}
Mit genügend vielen Einschränkungen versteht man das klassische Zweikörperproblem, wenn man die Lösung des Anfangswertproblems
\begin{equation}\label{eom0}
\ddot{\vec{x}} = - \frac{k}{m} \frac{\vec{x}}{|\vec{x}|^3},~\vec{x}(t_0) = \vec{x}_0,~\dot{\vec{x}} (t_0) = \vec{v}_0
\end{equation}
mit $m\in \mathbb{R}^{>0}$ und $k\in \mathbb{R}\setminus\{0\}$ versteht.
\subsection{Potential der Bewegungsgleichung}
  Wenn man die Gleichung \eqref{eom0} sehr lange anschaut, dann stellt man fest, dass man die rechte Seite als Gradient schreiben kann. Betrachtet man nämlich die Funktion
  \[
  \phi:\rr^n\oo \to \rr, \vec{x} \mapsto - \cmc \cdot \frac{1}{|\vec{x}|},
  \]
  dann stellt man fest, dass gilt
  \[
  \nabla \phi = -\cmc \nabla \left(\frac{1}{|\vec{x}|}\right) = \cmc \frac{\vec{x}}{|\vec{x}|}.
  \]
  Wir können also \eqref{eom0} schreiben als
  \begin{equation}\label{eompot}
    \ddot{\vec{x}} = -\nabla \phi
  \end{equation}
  Wir nennen $\phi$ das \emph{Potential} der Gleichung \eqref{eom0}.
\subsection{Energie und Impulserhaltung}
  Für jede Lösung $\vec{x}:\mathbb{R}\to \mathbb{R}^3$ von \eqref{eom0} betrachten wir die Funktion
  \[
  E:\mathbb{R} \to \mathbb{R},~t\mapsto \frac{1}{2}m\langle\dot{x}(t),\dot{x}(t)\rangle + m \phi(\vec{x}(t)).
  \]
  Wir nennen $E$ die \emph{Energie} der Bewegung. Wir rechnen jetzt nach, dass $E$ zeitlich konstant ist. Dazu nutzten wir, dass wir die Bewegungsgleichung auch in der Form \label{eompot} schreiben können.
  \begin{align*}
    \frac{dE}{dt} &= \frac{d}{dt}\left(\frac{1}{2}m\langle\dot{x}(t),\dot{x}(t)\rangle \right) + \frac{d}{dt}\left(m\phi(\vec{x}(t))\right)\\
     &= m \langle \dot{\vec{x}},\ddot{\vec{x}}\rangle + m \langle \nabla \phi , \dot{\vec{x}}\rangle\\
    &= m \langle \dot{\vec{x}}, \ddot{\vec{x}} + \nabla \phi\rangle\\
    &\overset{\eqref{eompot}}{=} m \langle \dot{\vec{x}}, -\nabla \phi + \nabla \phi \rangle \\
    &= m \langle \dot{\vec{x}},0 \rangle = 0
  \end{align*}
  Weiterhin betrachten wir die Funktion
  \[
  \vec{L}:\mathbb{R}^3\to \mathbb{R}^3, \vec{x}\mapsto m\left(\vec{x}\times \dot{\vec{x}}\right).
  \]
  Wir nennen $\vec{L}$ den \emph{Drehimpuls} der Bewegung. Auch $\vec{L}$ ist zetilich konstant. Dazu nutzen wir aus, dass für beliebige $\lambda \in \mathbb{R}$ und $\vec{a}\in \mathbb{R}$ die Beziehung $\lambda\vec{a}\times \vec{a}=0$ gilt.
  \begin{align*}
  \frac{\partial \vec{L}}{\partial t}&= m \frac{\partial}{\partial t}\left(\vec{x}\times \dot{\vec{x}}\right)\\
  &= m \left(\frac{\partial \vec{x}}{\partial t}\times \dot{\vec{x}} + \vec{x}\times \frac{\partial \dot{\vec{x}}}{\partial t} \right)\\
  &= m \left(\dot{\vec{x}}\times \dot{\vec{x}} + \vec{x}\times \ddot{\vec{x}} \right)\\
  &\overset{\eqref{eom0}}= m \left(0 + \vec{x}\times \frac{-k}{m|\vec{x}|^3}\vec{x}\right)\\
  &= 0
  \end{align*}
