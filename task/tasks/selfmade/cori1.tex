\begin{minipage}[b]{0.75\textwidth}
\begin{Exercise}[difficulty = 3, origin = Orpheus e.V, title = Fluß, label = cori1]
	Wir befinden uns auf einer geographischen Breite von $15^\circ$ auf der Nordhalbkugel, und betrachten einen Fluß, der direkt von Norden nach Süden fließt.\\ Der Fluß hat eine Fließgeschwindigkeit von $2~\mathrm{m\cdot s^{-1}}$, und eine Breite von $b = 1000~\mathrm{m}$. Wie groß ist der Pegelunterschied zwischen dem westlichen und östlichen Flußufer, wenn sich die Flußoberfläche so einstellt, dass sie senkrecht zur wirkenden Gesamtkraft steht. Die Winkelgeschwindigkeit der Erde beträgt $\omega \approx 7.29~\mathrm{s^{-1}}$.
\end{Exercise}
\end{minipage}
\hfill
\begin{minipage}[b]{0.2\textwidth}
	\tikzset{->-/.style={decoration={
			markings,
			mark=at position .5 with {\arrow{>}}},postaction={decorate}}}
\begin{tikzpicture}
\node[draw,ellipse,minimum width=4cm,minimum height=2cm] (ell) {};
\foreach \ang in {-80,-20,-10,0,10,20,80} {
  \pgfmathtruncatemacro{\rang}{180 - \ang}
  \draw[very thin, dashed,shorten >=1pt,shorten <=1pt] (ell.\ang) -- (ell.\rang);
}
\foreach \ang in {-65,-45,...,65} {
  \pgfmathsetmacro{\xrad}{2*sin(\ang)}
  \draw[thin, dashed] (0,1) arc [x radius=\xrad,y radius=1,start angle=90,end angle=-90];
}
\pgfmathsetmacro{\xrad}{2*sin(10)}
\draw[very thick, blue,->-] (0,1) arc[x radius = \xrad, y radius = 1, start angle = 90, end angle = -90];
  \pgfmathtruncatemacro{\rang}{180 - 15}
  \draw[thick, shorten >=1pt,shorten <=1pt,] (ell.15) -- (ell.\rang);
 \node at (2.25,.65) {$15^\circ~\mathrm{N}$};

\end{tikzpicture}
\end{minipage}
