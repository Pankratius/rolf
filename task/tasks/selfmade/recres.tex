\begin{Exercise}[title = Rekursives Widerstandsnetz, origin = Kurzaufgabe?, label = recres]
	Gegeben sei das unten abgebildete Widerstandsnetz, welches jeweils genau $n$ parallele Seiten in den äußeren Umrandungen hat. Jeder Kantenwiderstand beträgt $r$.
	\begin{figure}[h]
		\centering
		\begin{circuitikz}
			\ctikzset{bipoles/resistor/height=0.15}
			\ctikzset{bipoles/resistor/width=0.4}
			\draw (-1.5,1.5) to[short,o-] (-1,1);
			\node at (-1.75,1.5) {$A$};  
			\node at (5.75,-5.5) {$B$};
			\draw (5.5,-5.5)to [short,o-] (5,-5) ;;
			\draw (0,0) to [R] (4,0) to [R] (4,-4) to [R] (0,-4)  to [R] (0,0) to [R] (1,-1) to [R] (3,-1) to [R] (4,0);
			\draw (1,-1) to [R] (1,-3) to [R] (0,-4);
			\draw (1,-3) to [R] (3,-3) to [R] (4,-4);
			\draw (3,-3) to [R] (3,-1);
			\node at (2,-2) {$...$};
			\draw (-1,1) to [R] (5,1) to [R] (5,-5) to [R] (-1,-5) to [R] (-1,1);
			\draw (-1,1) to [R] (0,0);
			\draw (5,1) to [R] (4,0);
			\draw (5,-5) to [R] (4,-4);
			\draw (-1,-5) to [R] (0,-4);
		\end{circuitikz}
		\caption{Widerstandsnetz mit $n=3$}
		\label{fig:regres:orig}
	\end{figure}\\
	\sout{Berechne den Widerstand zwischen den Punkten $A$ und $B$ in Abhängigkeit der Tiefe $n$.} Berechne den Widerstand zwischen den Punkten $A$ und $B$ für ein unendlich großes Widerstandnetz.
\end{Exercise}
\begin{Answer}[ref = recres]
	\begin{figure}[h]
		\centering
	\begin{tikzpicture}
	\ctikzset{bipoles/resistor/height=0.1}
	\ctikzset{bipoles/resistor/width=0.3}
		\draw (-.5,.5) to[short,o-] (0,0);
		\draw (0,0) to [R] (3,0) to [R] (3,-3) to [R] (0,-3) to [R] (0,0);
		\draw (0,0) to [R] (0.5,-0.5) to [R] (2.5,-0.5) to [R] (2.5,-2.5) to [R] (0.5,-2.5) to [R] (0.5,-0.5);
		\draw[red] (0,-3) to [R, color = red,*-*] (0.5,-2.5);
		\draw (3,-3) to [R] (2.5,-2.5);
		\draw[red] (3,0) to [R,color = red, *-* ] (2.5,-.5);
		\draw (3,-3) to [short,-o] (3.5,-3.5);
		\node at (-1,.5) {$A$};
		\node at (4,-3.5) {$B$}; 
		\draw[dotted, thin] (-1,1) to (4,-4);
		\draw[dashed, thin] (-1,-4) to (4,1);
		\node[below, left] at (0,-3) {$\varphi_C$};
		\node[above, right] at (3,0) {$\varphi_C$};
		\node[below left = 2.5pt of {(2.5,-.5)}] {$\varphi_D = \varphi_C$};
		\node[above right = 2.5pt of {(.5,-2.5)}] {$\varphi_D=\varphi_D$};

	\end{tikzpicture}
	\caption{Spiegelsymmetrie im Stromkreis}
	\label{fig:recres:sym}
	\end{figure}
	Der Stromkreis ist spiegelsymmetrisch zur Achse durch $A$ und $B$. Also liegen Punkte, die durch Spiegelung an dieser Achse in einander übergehen auf dem gleichen Potenzial. \\
	Daraus folgt insbesondere, dass Punkte, die auf der Gerade senkrecht zu der durch $A$ und $B$ liegen, alle das gleiche Potential haben. Die jeweils anliegenden Kantenwiderstände können also getrennt werden.\\
	Damit vereinfacht sich der in Abbildung \ref{fig:regres:orig} gezeigte Stromkreis zu dem in Abbildung \ref{fig:recres:sym}.
	\begin{figure}[h]
		\begin{subfigure}[t]{.5\textwidth}
				\begin{circuitikz}[scale = .8]
					\ctikzset{bipoles/resistor/height=0.15}
					\ctikzset{bipoles/resistor/width=0.4}
					\draw (-1.5,1.5) to[short,o-] (-1,1);
					\node at (-1.75,1.5) {$A$};  
					\node at (5.75,-5.5) {$B$};
					\draw (5.5,-5.5)to [short,o-] (5,-5) ;;
					\draw (0,0) to [R,o-] (4,0) to [R] (4,-4) to [R] (0,-4)  to [R] (0,0) to [R] (1,-1) to [R] (3,-1);
					\draw (1,-1) to [R] (1,-3);
					\draw (1,-3) to [R] (3,-3) to [R] (4,-4);
					\draw (3,-3) to [R] (3,-1);
					\node at (2,-2) {$...$};
					\draw (-1,1) to [R] (5,1) to [R] (5,-5) to [R] (-1,-5) to [R] (-1,1);
					\draw (-1,1) to [R] (0,0);
					%\draw (5,1) to [R] (4,0);
					\draw (5,-5) to [R,-o] (4,-4);
					\draw[purple,very thick, rounded corners] (-.2,.2) rectangle (4.2,-4.2);
					\node at (.5,-4.5) {$R_n$};
					\draw[green, very thick, rounded corners,dashed] (-1.2,1.2) rectangle (5.2,-5.2);
					\node at (.5,-5.5) {$R_{n+1}$};
					
				\end{circuitikz}
				\caption{vereinfachter Stromkreis der Tiefe $n+1$}
				\label{fig:recres:simple}
		\end{subfigure}
		\begin{subfigure}[t]{.5\textwidth}
				\begin{circuitikz}[scale = .8]
					\ctikzset{bipoles/resistor/height=0.15}
					\ctikzset{bipoles/resistor/width=0.4}
					\draw (-1.5,1.5) to[short,o-] (-1,1);
					\node at (-1.75,1.5) {$A$};  
					\node at (5.75,-5.5) {$B$};
					\draw (5.5,-5.5)to [short,o-] (5,-5) ;
					\draw (-1,1) to [R] (5,1) to [R] (5,-5) to [R] (-1,-5) to [R] (-1,1);
					\draw (-1,1) to [R,-o] (0,0);
					\draw (0,0) to [R] (4,-4);
					\node[rotate = -45, above] at (2.2,-2.2) {$R_n$};
					\draw (5,-5) to [R,-o] (4,-4);
					\draw[purple,very thick, rounded corners] (-.2,.2) rectangle (4.2,-4.2);
					\node at (.5,-4.5) {$R_n$};
					\draw[green, very thick, rounded corners,dashed] (-1.2,1.2) rectangle (5.2,-5.2);
					\node at (.5,-5.5) {$R_{n+1}$};
			\end{circuitikz}
			\caption{Ersatzschaltung für Tiefe von $n+1$}
			\label{fig:recres:reallysimple}
		\end{subfigure}
		\caption{vereinfachte Stromkreise}
		\label{fig:recres:simpler}
	\end{figure}
	In Abildung \ref{fig:recres:simple} sieht man gut, wie sich der Stromkreis der Tiefe $n+1$ aus dem der Tiefe $n$ erzeugen lässt. Dazu wird an die Ecken des Stromkreises der Tiefe $n$ auf der Symmetrieachse jeweils ein Widerstand $R$ angelegt, und der Stromkreis dann durch ein äußeres Widerstandsquadrat mit den Punkten $A$ und $B$ verbunden.\\
	Wenn wir den Ersatzwiderstand des Stromkreises der Tiefe $n$ mit $R_n$ bezeichnen, dann ergibt sich ein Stromkreis, wie in Abbildung \ref{fig:recres:reallysimple} gezeigt. Dieser ist aber eine einfache Kombination aus Reihen und Parallelschaltungen, weswegen wir für den Ersatzwiderstand bei einer Tiefe $n+1$ schreiben können
	\begin{equation}\label{recres:receq}
	\frac{1}{R_{n+1}} = \frac{1}{R_n+2r} + \frac{1}{R} \implies R_{n+1} = \frac{R\left(2r+R_n\right)}{3r+R_n}.
	\end{equation}
	Damit der Widerstand bei der Tiefe $n$ eindeutig bestimmt ist, müssen wir noch $R_1$ angeben, was in diesem Fall $R_1 = R$ ist. \\
	Im Grenzfall einer unendlichen Tiefe ist $R_n = R_{n+1}=: \tilde{R}$, weil das Entfernen einer Widerstandsebene den Gesamtwiderstand nicht ändern sollte.\\
	Damit ergibt \eqref{recres:receq}
	\begin{align*}
		 \tilde{R}  &= \frac{r\left(2r+\tilde{R}\right)}{3r+\tilde{R}}\\
		\Leftrightarrow 3r\tilde{R} + \tilde{R}^2 &= 2r^2 + R\tilde{R}\\
		\Leftrightarrow 0&=\tilde{R}^2 + 2r\tilde{R} - 2r^2 \\
		\Aboxed{\implies \tilde{R} &= r\left(\sqrt{3}-1\right).}
	\end{align*}
	Die negative Lösung kann ausgeschlossen werden, weil Widerstände nicht negativ sind.\\
	\noindent\rule{\textwidth}{.75pt}
	\textit{Bemerkung:} \\
	Es scheint, als ließe sich die Konvergenz der Folge $\left(R_n\right)_{n\in \mathbb{N}}$ am Besten durch ein explizietes Lösen der Rekursionsgleichung zeigen.
	Definiere dazu $S_n := 2r + R_n \Leftrightarrow S_n - 2r = R_n$. Einsetzen in \eqref{recres:receq} führt auf 
	\begin{equation}\label{recres:intermediate}
		S_{n+1} - 2r = r\frac{S_n}{S_n+r} = \frac{S_n}{1+\nicefrac{S_n}{r}}.
	\end{equation}
	Definiere $T_n := 1+\frac{S_n}{r} \Leftrightarrow r\left(T_n-1\right) = S_n$. Einsetzen in \eqref{recres:intermediate} führt auf 
	\begin{align}\label{recres:teq}
		r\left(T_{n+1}-1\right) - 2r &= \frac{r\left(T_n-1\right)}{T_n}\notag\\
		\Leftrightarrow T_{n+1}-3 &= \frac{T_n-1}{T_n} = 1 -\frac{1}{T_n}\notag\\
		\Leftrightarrow T_{n+1} &= 4 - \frac{1}{T_n}.
	\end{align}
	Wir versuchen nun, \eqref{recres:teq} auf eine Form zu bringen, die wir Lösen können. Betrachte dazu eine Folge\footnote[2]{Brand, Louis: \glqq A sequence defined by a difference equation", American Mathematical Monthly 62, September 1955, S.489-492} $\left(x_n\right)_{n\in \mathbb{N}}$ mit $T_n = \frac{x_{n+1}}{x_n}$. Einsetzen in \eqref{recres:teq} führt auf 
	\begin{equation}
		\frac{x_{n+2}}{x_{n+1}} = 4 - \frac{x_n}{x_{n+1}} \Rightarrow x_{n+2} - 4x_{n+1} + x_{n} = 0.
	\end{equation}
	Diese Gleichung kann durch den Ansatz $x_n = a\cdot \lambda^n$ gelöst werden ($a\neq 0$), und sollte zwei Lösungen für $\lambda$ liefern. Die allgemeine Lösung ist dann eine Superposition der beiden. \footnote[3]{K.F.Riley, M.P. Hobson, S.J.Bence: \glqq Mathematical Methods for Physics and Engineering", Cambridge University Press, S.499}
	Einsetzen des Ansatzes führt auf $\lambda^2 - 4 \lambda + 1 = 0 \Leftrightarrow \lambda = 2\pm \sqrt{3}$. Also ist 
	\begin{align*}
		x_n &= c_1 \left(2-\sqrt{3}\right)^n + c_2 \left(2+\sqrt{3}\right)^n\\
		\Rightarrow T_n &= \frac{\left(2-\sqrt{3}\right)^{n+1} + c'\left(2+\sqrt{3}\right)^{n+1}}{\left(2-\sqrt{3}\right)^{n} + c'\left(2+\sqrt{3}\right)^{n}}\\
		\Leftrightarrow R_n &= r\left[\frac{\left(2-\sqrt{3}\right)^{n+1} + c'\left(2+\sqrt{3}\right)^{n+1}}{\left(2-\sqrt{3}\right)^{n} + c'\left(2+\sqrt{3}\right)^{n}}-3\right],
	\end{align*}
	mit $c_1,c_2\in \mathbb{R}$, o.B.d.A $c_1\neq 0$, $c':= \nicefrac{c_2}{c_1}$.\\
	Wir können nun den Grenzwert $\tilde{R}$ für $n\to\infty$ berechnen:
	\begin{align*}
		R_{n} &= r\left[\frac{\left(2-\sqrt{3}\right)^{n+1} + c'\left(2+\sqrt{3}\right)^{n+1}}{\left(2-\sqrt{3}\right)^{n} + c'\left(2+\sqrt{3}\right)^{n}}-3\right]\\
		&= r\left[\frac{\left(2-\sqrt{3}\right) \cdot \left(2-\sqrt{3}\right)^n + c'\cdot \left(2+\sqrt{3}\right)\cdot \left(2+\sqrt{3}\right)^n}{\left(2-\sqrt{3}\right)^n + c' \left(2+\sqrt{3}\right)}-3\right]\\
		&= r\left[\frac{\left(2-\sqrt{3}\right) \cdot \left(\frac{2-\sqrt{3}}{2+\sqrt{3}}\right)^n + c'\cdot \left(2+\sqrt{3}\right)}{\left(\frac{2-\sqrt{3}}{2+\sqrt{3}}\right)^n + c' }-3\right]\\
	\Rightarrow \tilde{R} =  \lim_{n\to \infty} R_n &= r\cdot \lim_{n\to \infty}\left[\frac{\left(2-\sqrt{3}\right) \cdot \left(\frac{2-\sqrt{3}}{2+\sqrt{3}}\right)^n + c'\cdot \left(2+\sqrt{3}\right)}{\left(\frac{2-\sqrt{3}}{2+\sqrt{3}}\right)^n + c' }-3\right]
	\end{align*}
	Weil $0<\frac{2-\sqrt{3}}{2+\sqrt{3}}<1$, ist $\lim_{n\to \infty}\left(\frac{2-\sqrt{3}}{2+\sqrt{3}}\right)^n = 0$, und damit
	\begin{equation*}
	\boxed{
		\tilde{R} = \lim_{n \to \infty} R_n = r\cdot \frac{c'\left(2+\sqrt{3}\right)}{c'} - 3 = r\cdot \left( \sqrt{3}-1\right).
		}
	\end{equation*}
	\\
	\noindent\rule{\textwidth}{.75pt}
	\textit{Bewertungsvorschlag:}\\
	Das mathematische korrekte Begründen der Konvergenz sollte man vermutlich nicht von den Teilnehmern erwarten. Es kann wohl eine kurze Begründung ausreichend sein.\\
	\begin{table}[h]
		\centering
\begin{tabular}{|c|c|}
	\hline 
	&Punkte  \\ 
	\hline 
	Erkennen der Spiegelsymmetrie&0.5  \\ 
	\hline 
	Vereinfachen des Stromkreises&0.5  \\ 
	\hline 
	Aufstellen der rekursiven Darstellung \eqref{recres:receq}&1  \\ 
	\hline 
	(kurze) Begründung der Konvergenz&0.25 \\ 
	\hline 
	Feststellung, dass im Grenzfall $R_{n+1} = R_n$ gilt & 0.25\\\hline
	Bestimmung des Grenwiderstands $\tilde{R}$ & 1 \\
	\hline\hline
	\textit{Gesamtpunktzahl:} & 3.5\\
	\hline\hline
\end{tabular} 
\end{table}
\end{Answer}
