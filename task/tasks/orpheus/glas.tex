\begin{Exercise}[title = Schwimmende Glasschale, origin = {Lucas Rettenmeier, Orpheus e.V.}, difficulty = 3, label = glass]
	Eine dünnwandige, ideal wärmeleitende  zylindrische Glasschale der Höhe $H = 15~\mathrm{cm}$ schwimmt mit dem Boden nach unten auf einer Wasseroberfläche, und taucht dabei bis zur Hälfte in das Wasser ein.\\
	Bestimme, wie tief die Glasschale eintaucht, wenn man sie umdreht, und ins Wasser tut. Berechne außerdem, wie tief man sie ins Wasser tauchen muss, damit sie nicht mehr schwimmt, sondern untergeht.\\
	Die Wasserdichte ist $\rho = 1000~\mathrm{\nicefrac{kg}{m^3}}$ und der äußere Luftdruck beträgt $p_0 = 101 325~\mathrm{Pa}$. \\ Die eingeschlossene Luft kann man näherungsweise als ein sog. \textit{ideales Gas} betrachten. Für ein solches gilt in diesem Fall die \textit{ideale Gasgleichung}, die besagt, dass das Verhältnis
			$\frac{pV}{T}$
	die ganze Zeit konstant bleibt. Dabei ist $p$ der Druck der Luft in dem Gefäß, $V$ das Volumen der eingeschlossenen Luft und $T$ die Temperatur. Gleichzeitig kann angenommen werden, dass sich die Lufttemperatur während des Umdrehens nicht ändert. 
\end{Exercise}