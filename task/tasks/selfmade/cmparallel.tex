\begin{Exercise}[label = cmparallel, difficulty = 3, origin = P. Gnädig, title = Parallellandung]
	Wir betrachten eine Rakete die von der Erde gestartet wird, und deren Geschwindigkeitsvektor bei der Landung parallel zum Geschwindigkeitsvektor beim Start war. Der Start- und Landepunkt schließen einen Winkel $\theta$ mit dem Erdmittelpunkt ein. Fliegt die Rakete direkt über der Erde in einer Kreisbahn, dann benötigt sie dafür die Zeit $T_0$. \\
	Wie lange benötigt die Rakete für diesen besonderen Flug, und wie groß ist dabei die maximale Höhe, die sie erreichen kann?
\end{Exercise}
\begin{Answer}[ref = cmparalle]
	Wenn die Geschwindigkeitsvektoren bei Start und Landung parallel sein müssen, ist dies, bei gegebener elliptischer Bahnform (1. keplersches Gesetz) nur dann möglich, wenn die Raketenbahn die Erde an den Punkten schneidet, die senkrecht über dem Ellipsenmittelpunkt sind. Folglich entspricht der Abstand zwischen Start- und Landepunkt der kleinen Halbachse der Ellipsenbahn.\\
	 
\end{Answer}