\begin{Exercise}[label = grassh, title = fauler Grasshüpfer, origin = P.Gnädig, difficulty = 5]
	Ein fauler Grasshüpfer möchte über einen Baumstumpf mit Radius $r = 20~\mathrm{cm}$ springen.\\
	Wie groß ist die dafür mindestens benötigte Geschwindigkeit, wenn der Luftwiderstand vernachlässigt werden kann?
\end{Exercise}
\begin{Answer}
	Bei minimaler Anfangsgeschwindigkeit sollte die Grashüpferbahn den Baumstamm in genau zwei Punkten symmetrisch zur Achse um den Baumstumpf berühren, und damit ihren Scheitelpunkt direkt auf dieser Achse haben, wie in Abbildung \ref{fig:grasshs} gezeigt ist.\\
	Es ist jetzt am einfachsten, wenn man zuerst die Bewegung des Grashüpfers von $ B$ nach $B^{\ast}$ betrachtet. Die Geschwindigkeit des Grashüpfers in $B$ nennen wir $v_2$, und den Winkel, den die Grashüpfergeschwindigkeit mit dem Boden macht $\theta$.\\
	Dann wissen wir, dass die Wurfweite $s_w$ dieses schrägen Wurfes gegeben ist durch
	\begin{equation}\label{grassh:weite}
		s_w = \frac{2 v_2^2 \sin \theta \cos \theta}{g}.
	\end{equation}
	Geometrisch interpretiert muss diese Wurfweite $s_w$ genau die Strecke zwischen $B$ und $B^{\ast}$ sein. Wir versuchen jetzt, $\overline{BB^{\ast}}$ durch bekannte Parameter auszudrücken.\\
	Dazu macht es Sinn, wenn wir uns den Winkel $\angle COB$ anschauen. Wir wissen nämlich, dass die Geschwindigkeit 
\end{Answer}
\begin{figure}[h]
	\centering
	\begin{tikzpicture}[line cap=round,line join=round,>=triangle 45,x=1.0cm,y=1.0cm]
\clip(-3.573888715158432,-0.6284321850818549) rectangle (3.831338230521266,5.4106580654465875);
\draw [shift={(-3.,0.)}] (0,0) -- (0.:0.5107053065985999) arc (0.:72.38196924975453:0.5107053065985999) -- cycle;
\draw [shift={(-1.6166086789302696,3.1775297784800722)}] (0,0) -- (0.:0.5107053065985999) arc (0.:53.13793429316053:0.5107053065985999) -- cycle;
\draw [shift={(0.,2.)}] (0,0) -- (90.:0.5107053065985999) arc (90.:143.930590100419:0.5107053065985999) -- cycle;
\draw(0.,2.) circle (2.cm);
\draw [rotate around={90.:(0.,-25.3395997404539)}] (0.,-25.3395997404539) ellipse (29.719599740453898cm and 5.741326023857797cm);
\draw [->] (-1.6166086789302696,3.1775297784800722) -- (-0.76,4.32);
\draw (-1.6166086789302696,3.1775297784800722)-- (1.61660867893027,3.1775297784800722);
\draw (-1.6166086789302696,3.1775297784800722)-- (0.,2.);
\draw [->] (-3.,0.) -- (-2.6035486326210924,1.2484098166679993);
\begin{scriptsize}
\draw [fill=black] (-3.,0.) circle (2pt);
\draw [fill=black] (3.,0.) circle (2pt);
\draw [fill=black] (0.,2.) circle (2pt);
\draw [fill=black] (0.,0.) circle (2pt);
\draw [fill=black] (-1.6166086789302696,3.1775297784800722) circle (2pt);
\draw [fill=black] (1.61660867893027,3.1775297784800722) circle (2pt);
\draw [fill=black] (0.,4.38) circle (2pt);
%\draw [fill=black] (-0.76,4.32) circle (2.5pt);
\draw [fill=black] (0.,3.1775297784800722) circle (2pt);
%\draw [fill=black] (-2.6035486326210924,1.2484098166679993) circle (2.5pt);
\end{scriptsize}
\draw[thick] (-3.5,0)--(3.5,0);
\draw[thin,dashed](0,0)--(0,5);
\fill[pattern = north east lines] (-3.5,0)rectangle(3.5,-0.5);
\node at (-3.2,0.2){$A$};
\node at (3.2,0.2){$A^{\ast}$};
\node at (-2,3.1775297784800722){$B$};
\node at (2,3.1775297784800722){$B^{\ast}$};
\node at (0.5,2){$O$};
\node at (-1.25,2.5){$r$};
\node at (-2.25,0.2){$\alpha$};
\node at (-0.5,2.7) {$\theta$};
\node at (-0.9,3.4){$\theta$};
\node at (-1.8,3.8){$v_2$};
\node at (0.4,3.4){$C$};
\end{tikzpicture}
	\caption{Skizze der Grashüpferbahn}
	\label{fig:grasshs}		
\end{figure}