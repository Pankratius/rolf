\begin{Exercise}[label = Schicksal der Erde, title = Schicksal der Erde, difficulty = 3, origin = 1. Runde zur 46. IPhO 2015]
Im Laufe ihrer Entwicklung verändert sich die Zusammensetzung der Sonne durch die in ihrem
Inneren stattfindenden Fusionsprozesse. Bis zum Ende ihrer Phase als Hauptreihenstern wird
der Sonnenradius dadurch auf etwa das 1,6-fache des jetzigen Wertes ansteigen, während ihre
Oberflächentemperatur auf etwa 96 \% des heutigen Wertes sinkt.
Beim jetzigen Entwicklungsstand der Sonne würde sich ohne Berücksichtigung des Treibhauseffektes auf der Erde eine Gleichgewichtstemperatur von etwa
246
K einstellen.
Schätze ab, um wie viel sich diese Temperatur durch die Veränderung der Sonne verschieben
wird. Erläutere kurz, was dies für das Leben auf der Erde bedeuten könnte.
Nimm für die Abschätzung an, dass sich der Bahnradius und andere relevante Parameter der
Erde nicht verändern und die Temperatur auf der gesamten Erde die gleiche ist
\end{Exercise}