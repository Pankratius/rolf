\begin{Exercise}[label = Wärmekraftmaschine, origin = {4. Runde zur 46. IPhO 2015}, title = Wärmekraftmaschine, difficulty = 3]
Gegeben seien die Punkte $A(V_0,p_0)$, $B(\frac{3}{2}V_0,\frac{5}{2}	p_0)$, $C(3V_0,p_0)$ im p-V-Diagramm. In einer Wärmekraftmaschine durchläuft eine Stoffmenge $n$ eines idealen Gases diese Punkte. Dabei wird Wärme aus einem Reservoir der Temperatur $T_h$ aufgenommen und Wärme in ein Reservoir der Temperatur $T_k$ abgegeben.
\begin{enumerate}
\item Was ist der Wirkungsgrad?
\item Welcher Bedingung müssen $T_h$ und $T_k$ genügen, damit dieser Prozess ablaufen kann?
\end{enumerate}      	
\end{Exercise}
