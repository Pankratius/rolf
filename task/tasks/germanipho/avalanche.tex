\begin{Exercise}[label = avalance, origin = {4. Runde IPhO, 2008}, title = {Lawine}, difficulty = 4]
	Auf einem Hang mit Neigungswinkel $\alpha$ liegen in konstantem Abstand $d$ identische Schneeflocken der Masse $m$. Die obere Flocke wird leicht angeschoben,
sodass eine Lawine entsteht. Der Aufschlag der Lawine auf die ruhenden Flocken ist immer völlig
inelastisch, die Gleitreibung kann vernachlässigt werden. Welche Beschleunigung hat die Lawine
nachdem diese viele Flocken angesammelt hat?
\end{Exercise}
\begin{Answer}
	Der Trick bei der Lösung dieser Aufgabe besteht darin, dass man von den einzelnen, also diskreten Schneeflocken, zu einer kontinuirlichen übergeht.\\
	Wenn man das ganz brachial macht, dann geht das so:
	Wir bezeichnen mit $y\left(t\right)$ die Position der Lawine nach einer bestimmten Zeit, und stellen die Bewegungsgleichung auf. Dabei beachten wir, dass die zeitliche Änderung des Impuls gerade der wirkenden Kraft entspricht, und die Dichte $\lambda$ der Lawine konstant sein sollte. Jetzt können wir mit $p = mv = y \lambda v = y \lambda \dot{v}$ einfach schreiben
	\begin{equation}\label{avalanche:force}
		\frac{dp}{dt} = F \Rightarrow y \ddot{y} + \dot{y}^2 = yg \sin \alpha.
	\end{equation}
	Das ist eine Differentialgleichung, die wir in $y\left(t\right)$ lösen könnten. Dabei müssen wir uns über die Anfangsbedingungen wirklich gar keine Sorgen machen, weil uns nur der Grenzfall $\ddot{y}\left(t\right)$ für $t\rightarrow \infty$ interessiert. \\
	Schneller gehts, wenn wir uns zu nutze machen, dass die Aufgabenstellung schon suggeriert, dass die Lawine am Ende eine gleichmäßige Beschleunigung hat. Deswegen könnenw wir einfach mal $y\left(t\right) = \frac{1}{2}at^2$ einsetzen, und schauen, was passiert. Wir erhalten dann nämlich einfach
	\begin{equation}\label{avalance:acc}
	\boxed{
		\frac{1}{2} a^2 t^2 + a^2 t^2 = \frac{1}{2}at^2g\sin \alpha \Rightarrow a = \frac{1}{3}\sin \alpha.}
	\end{equation}
\end{Answer}